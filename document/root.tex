\documentclass[11pt,a4paper]{article}
\usepackage[T1]{fontenc}
\usepackage{isabelle,isabellesym}

% further packages required for unusual symbols (see also
% isabellesym.sty), use only when needed

%\usepackage{amssymb}
  %for \<leadsto>, \<box>, \<diamond>, \<sqsupset>, \<mho>, \<Join>,
  %\<lhd>, \<lesssim>, \<greatersim>, \<lessapprox>, \<greaterapprox>,
  %\<triangleq>, \<yen>, \<lozenge>

%\usepackage{eurosym}
  %for \<euro>

%\usepackage[only,bigsqcap,bigparallel,fatsemi,interleave,sslash]{stmaryrd}
  %for \<Sqinter>, \<Parallel>, \<Zsemi>, \<Parallel>, \<sslash>

%\usepackage{eufrak}
  %for \<AA> ... \<ZZ>, \<aa> ... \<zz> (also included in amssymb)

%\usepackage{textcomp}
  %for \<onequarter>, \<onehalf>, \<threequarters>, \<degree>, \<cent>,
  %\<currency>

% this should be the last package used
\usepackage{pdfsetup}

% urls in roman style, theory text in math-similar italics
\urlstyle{rm}
\isabellestyle{it}

% for uniform font size
%\renewcommand{\isastyle}{\isastyleminor}


\begin{document}

\title{A non-Boolean Gray code}
\author{Maximilian Spitz}
\maketitle

\begin{abstract}
The original Gray code after Frank Gray, also known as reflected binary code (RBC), is an ordering of the binary numeral system such that two successive values differ only in one bit. We provide a theory for a non-Boolean Gray code, which is a generalisation of the idea for an arbitrary base.
Contained is the necessary theoretical environment to express and reason about the respective properties.
\end{abstract}

\tableofcontents

% sane default for proof documents
\parindent 0pt\parskip 0.5ex

% generated text of all theories
%
\begin{isabellebody}%
\setisabellecontext{Encoding{\isacharunderscore}{\kern0pt}Nat}%
%
\isadelimdocument
%
\endisadelimdocument
%
\isatagdocument
%
\isamarkupsection{An Encoding for Natural Numbers%
}
\isamarkuptrue%
%
\endisatagdocument
{\isafolddocument}%
%
\isadelimdocument
%
\endisadelimdocument
%
\isadelimtheory
%
\endisadelimtheory
%
\isatagtheory
\isacommand{theory}\isamarkupfalse%
\ Encoding{\isacharunderscore}{\kern0pt}Nat\isanewline
\ \ \isakeyword{imports}\ Main\isanewline
\isakeyword{begin}%
\endisatagtheory
{\isafoldtheory}%
%
\isadelimtheory
%
\endisadelimtheory
%
\begin{isamarkuptext}%
At first, an encoding of naturals as lists of digits with respect to
  an arbitrary base $b \geq 2$ is introduced because the presented
  Gray code and its properties are reasonably expressed in terms of
  a word representation of numbers.%
\end{isamarkuptext}\isamarkuptrue%
%
\isadelimdocument
%
\endisadelimdocument
%
\isatagdocument
%
\isamarkupsubsection{Validity and Valuation%
}
\isamarkuptrue%
%
\endisatagdocument
{\isafolddocument}%
%
\isadelimdocument
%
\endisadelimdocument
%
\begin{isamarkuptext}%
In the context of a given base, not all possible code words are valid
  number representations.
A validity predicate is defined, that checks if a code word is valid
  and a valuation to obtain the number represented by a valid word.%
\end{isamarkuptext}\isamarkuptrue%
\isacommand{type{\isacharunderscore}{\kern0pt}synonym}\isamarkupfalse%
\ base\ {\isacharequal}{\kern0pt}\ nat\isanewline
\isanewline
\isacommand{type{\isacharunderscore}{\kern0pt}synonym}\isamarkupfalse%
\ word\ {\isacharequal}{\kern0pt}\ {\isachardoublequoteopen}nat\ list{\isachardoublequoteclose}\isanewline
\isanewline
\isacommand{fun}\isamarkupfalse%
\ val\ {\isacharcolon}{\kern0pt}{\isacharcolon}{\kern0pt}\ {\isachardoublequoteopen}base\ {\isasymRightarrow}\ word\ {\isasymRightarrow}\ nat{\isachardoublequoteclose}\ \isakeyword{where}\isanewline
\ \ {\isachardoublequoteopen}val\ b\ {\isacharbrackleft}{\kern0pt}{\isacharbrackright}{\kern0pt}\ {\isacharequal}{\kern0pt}\ {\isadigit{0}}{\isachardoublequoteclose}\isanewline
{\isacharbar}{\kern0pt}\ {\isachardoublequoteopen}val\ b\ {\isacharparenleft}{\kern0pt}a{\isacharhash}{\kern0pt}w{\isacharparenright}{\kern0pt}\ {\isacharequal}{\kern0pt}\ a\ {\isacharplus}{\kern0pt}\ b{\isacharasterisk}{\kern0pt}val\ b\ w{\isachardoublequoteclose}\isanewline
\isanewline
\isacommand{fun}\isamarkupfalse%
\ valid\ {\isacharcolon}{\kern0pt}{\isacharcolon}{\kern0pt}\ {\isachardoublequoteopen}base\ {\isasymRightarrow}\ word\ {\isasymRightarrow}\ bool{\isachardoublequoteclose}\ \isakeyword{where}\isanewline
\ \ {\isachardoublequoteopen}valid\ b\ {\isacharbrackleft}{\kern0pt}{\isacharbrackright}{\kern0pt}\ {\isasymlongleftrightarrow}\ {\isadigit{2}}{\isasymle}b{\isachardoublequoteclose}\isanewline
{\isacharbar}{\kern0pt}\ {\isachardoublequoteopen}valid\ b\ {\isacharparenleft}{\kern0pt}a{\isacharhash}{\kern0pt}w{\isacharparenright}{\kern0pt}\ {\isasymlongleftrightarrow}\ a{\isacharless}{\kern0pt}b\ {\isasymand}\ valid\ b\ w{\isachardoublequoteclose}%
\begin{isamarkuptext}%
Given a base, the value of a valid word is bound by its length.%
\end{isamarkuptext}\isamarkuptrue%
\isacommand{lemma}\isamarkupfalse%
\ val{\isacharunderscore}{\kern0pt}bound{\isacharcolon}{\kern0pt}\isanewline
\ \ {\isachardoublequoteopen}valid\ b\ w\ {\isasymLongrightarrow}\ val\ b\ w\ {\isacharless}{\kern0pt}\ b{\isacharcircum}{\kern0pt}length{\isacharparenleft}{\kern0pt}w{\isacharparenright}{\kern0pt}{\isachardoublequoteclose}\isanewline
%
\isadelimproof
%
\endisadelimproof
%
\isatagproof
\isacommand{proof}\isamarkupfalse%
\ {\isacharparenleft}{\kern0pt}induction\ w{\isacharparenright}{\kern0pt}\isanewline
\ \ \isacommand{case}\isamarkupfalse%
\ Nil\ \isacommand{thus}\isamarkupfalse%
\ {\isacharquery}{\kern0pt}case\ \isacommand{by}\isamarkupfalse%
\ simp\isanewline
\isacommand{next}\isamarkupfalse%
\isanewline
\ \ \isacommand{case}\isamarkupfalse%
\ {\isacharparenleft}{\kern0pt}Cons\ a\ w{\isacharparenright}{\kern0pt}\isanewline
\ \ \isacommand{hence}\isamarkupfalse%
\ IH{\isacharcolon}{\kern0pt}\ {\isachardoublequoteopen}{\isadigit{1}}{\isacharplus}{\kern0pt}val\ b\ w\ {\isasymle}\ b{\isacharcircum}{\kern0pt}length{\isacharparenleft}{\kern0pt}w{\isacharparenright}{\kern0pt}{\isachardoublequoteclose}\ \isacommand{by}\isamarkupfalse%
\ simp\isanewline
\ \ \isacommand{have}\isamarkupfalse%
\ {\isachardoublequoteopen}val\ b\ {\isacharparenleft}{\kern0pt}a{\isacharhash}{\kern0pt}w{\isacharparenright}{\kern0pt}\ {\isacharless}{\kern0pt}\ b{\isacharasterisk}{\kern0pt}{\isacharparenleft}{\kern0pt}{\isadigit{1}}{\isacharplus}{\kern0pt}val\ b\ w{\isacharparenright}{\kern0pt}{\isachardoublequoteclose}\ \isacommand{using}\isamarkupfalse%
\ Cons{\isachardot}{\kern0pt}prems\ \isacommand{by}\isamarkupfalse%
\ auto\isanewline
\ \ \isacommand{also}\isamarkupfalse%
\ \isacommand{have}\isamarkupfalse%
\ {\isachardoublequoteopen}{\isachardot}{\kern0pt}{\isachardot}{\kern0pt}{\isachardot}{\kern0pt}\ {\isasymle}\ b{\isacharasterisk}{\kern0pt}b{\isacharcircum}{\kern0pt}length{\isacharparenleft}{\kern0pt}w{\isacharparenright}{\kern0pt}{\isachardoublequoteclose}\ \isacommand{using}\isamarkupfalse%
\ IH\ mult{\isacharunderscore}{\kern0pt}le{\isacharunderscore}{\kern0pt}mono{\isadigit{2}}\ \isacommand{by}\isamarkupfalse%
\ blast\isanewline
\ \ \isacommand{also}\isamarkupfalse%
\ \isacommand{have}\isamarkupfalse%
\ {\isachardoublequoteopen}{\isachardot}{\kern0pt}{\isachardot}{\kern0pt}{\isachardot}{\kern0pt}\ {\isacharequal}{\kern0pt}\ b{\isacharcircum}{\kern0pt}length{\isacharparenleft}{\kern0pt}a{\isacharhash}{\kern0pt}w{\isacharparenright}{\kern0pt}{\isachardoublequoteclose}\ \isacommand{by}\isamarkupfalse%
\ simp\isanewline
\ \ \isacommand{finally}\isamarkupfalse%
\ \isacommand{show}\isamarkupfalse%
\ {\isacharquery}{\kern0pt}case\ \isacommand{by}\isamarkupfalse%
\ blast\isanewline
\isacommand{qed}\isamarkupfalse%
%
\endisatagproof
{\isafoldproof}%
%
\isadelimproof
\isanewline
%
\endisadelimproof
\isanewline
\isacommand{lemma}\isamarkupfalse%
\ valid{\isacharunderscore}{\kern0pt}base{\isacharcolon}{\kern0pt}\isanewline
\ \ {\isachardoublequoteopen}valid\ b\ w\ {\isasymLongrightarrow}\ {\isadigit{2}}{\isasymle}b{\isachardoublequoteclose}\isanewline
%
\isadelimproof
\ \ %
\endisadelimproof
%
\isatagproof
\isacommand{by}\isamarkupfalse%
\ {\isacharparenleft}{\kern0pt}induction\ w{\isacharparenright}{\kern0pt}\ auto%
\endisatagproof
{\isafoldproof}%
%
\isadelimproof
%
\endisadelimproof
%
\isadelimdocument
%
\endisadelimdocument
%
\isatagdocument
%
\isamarkupsubsection{Encoding Numbers as Words%
}
\isamarkuptrue%
%
\endisatagdocument
{\isafolddocument}%
%
\isadelimdocument
%
\endisadelimdocument
%
\begin{isamarkuptext}%
It was stated that not all code words are valid. Similarly, numbers do not
  have a unique word representation in general.
Therefore, it is reasonable to normalise representations with respect
  to either value or word length.
A normal representation w.r.t. value is without leading zeroes.
However, if the word length is fixed, numbers can be represented
  only up to an upper bound. Note that this bound is stated above.%
\end{isamarkuptext}\isamarkuptrue%
\isacommand{fun}\isamarkupfalse%
\ enc\ {\isacharcolon}{\kern0pt}{\isacharcolon}{\kern0pt}\ {\isachardoublequoteopen}base\ {\isasymRightarrow}\ nat\ {\isasymRightarrow}\ word{\isachardoublequoteclose}\ \isakeyword{where}\isanewline
\ \ {\isachardoublequoteopen}enc\ {\isacharunderscore}{\kern0pt}\ {\isadigit{0}}\ {\isacharequal}{\kern0pt}\ {\isacharbrackleft}{\kern0pt}{\isacharbrackright}{\kern0pt}{\isachardoublequoteclose}\isanewline
{\isacharbar}{\kern0pt}\ {\isachardoublequoteopen}enc\ b\ n\ {\isacharequal}{\kern0pt}\ {\isacharparenleft}{\kern0pt}if\ {\isadigit{2}}{\isasymle}b\ then\ n\ mod\ b{\isacharhash}{\kern0pt}enc\ b\ {\isacharparenleft}{\kern0pt}n\ div\ b{\isacharparenright}{\kern0pt}\ else\ undefined{\isacharparenright}{\kern0pt}{\isachardoublequoteclose}\isanewline
\isanewline
\isacommand{fun}\isamarkupfalse%
\ enc{\isacharunderscore}{\kern0pt}len\ {\isacharcolon}{\kern0pt}{\isacharcolon}{\kern0pt}\ {\isachardoublequoteopen}base\ {\isasymRightarrow}\ nat\ {\isasymRightarrow}\ nat{\isachardoublequoteclose}\ \isakeyword{where}\isanewline
\ \ {\isachardoublequoteopen}enc{\isacharunderscore}{\kern0pt}len\ {\isacharunderscore}{\kern0pt}\ {\isadigit{0}}\ {\isacharequal}{\kern0pt}\ {\isadigit{0}}{\isachardoublequoteclose}\isanewline
{\isacharbar}{\kern0pt}\ {\isachardoublequoteopen}enc{\isacharunderscore}{\kern0pt}len\ b\ n\ {\isacharequal}{\kern0pt}\ {\isacharparenleft}{\kern0pt}if\ {\isadigit{2}}{\isasymle}b\ then\ Suc{\isacharparenleft}{\kern0pt}enc{\isacharunderscore}{\kern0pt}len\ b\ {\isacharparenleft}{\kern0pt}n\ div\ b{\isacharparenright}{\kern0pt}{\isacharparenright}{\kern0pt}\ else\ undefined{\isacharparenright}{\kern0pt}{\isachardoublequoteclose}\isanewline
\isanewline
\isacommand{fun}\isamarkupfalse%
\ lenc\ {\isacharcolon}{\kern0pt}{\isacharcolon}{\kern0pt}\ {\isachardoublequoteopen}nat\ {\isasymRightarrow}\ base\ {\isasymRightarrow}\ nat\ {\isasymRightarrow}\ word{\isachardoublequoteclose}\ \isakeyword{where}\isanewline
\ \ {\isachardoublequoteopen}lenc\ {\isadigit{0}}\ {\isacharunderscore}{\kern0pt}\ {\isacharunderscore}{\kern0pt}\ {\isacharequal}{\kern0pt}\ {\isacharbrackleft}{\kern0pt}{\isacharbrackright}{\kern0pt}{\isachardoublequoteclose}\isanewline
{\isacharbar}{\kern0pt}\ {\isachardoublequoteopen}lenc\ {\isacharparenleft}{\kern0pt}Suc\ k{\isacharparenright}{\kern0pt}\ b\ n\ {\isacharequal}{\kern0pt}\ n\ mod\ b{\isacharhash}{\kern0pt}lenc\ k\ b\ {\isacharparenleft}{\kern0pt}n\ div\ b{\isacharparenright}{\kern0pt}{\isachardoublequoteclose}\isanewline
\isanewline
\isacommand{definition}\isamarkupfalse%
\ normal\ {\isacharcolon}{\kern0pt}{\isacharcolon}{\kern0pt}\ {\isachardoublequoteopen}base\ {\isasymRightarrow}\ word\ {\isasymRightarrow}\ bool{\isachardoublequoteclose}\ \isakeyword{where}\isanewline
\ \ {\isachardoublequoteopen}normal\ b\ w\ {\isasymequiv}\ enc{\isacharunderscore}{\kern0pt}len\ b\ {\isacharparenleft}{\kern0pt}val\ b\ w{\isacharparenright}{\kern0pt}\ {\isacharequal}{\kern0pt}\ length\ w{\isachardoublequoteclose}%
\isadelimdocument
%
\endisadelimdocument
%
\isatagdocument
%
\isamarkupsubsection{Correctness%
}
\isamarkuptrue%
%
\endisatagdocument
{\isafolddocument}%
%
\isadelimdocument
%
\endisadelimdocument
%
\begin{isamarkuptext}%
Now, the expected properties of above definitions are proven as well as
  that they interact correctly.%
\end{isamarkuptext}\isamarkuptrue%
\isacommand{lemma}\isamarkupfalse%
\ length{\isacharunderscore}{\kern0pt}enc{\isacharcolon}{\kern0pt}\isanewline
\ \ {\isachardoublequoteopen}{\isadigit{2}}{\isasymle}b\ {\isasymLongrightarrow}\ length\ {\isacharparenleft}{\kern0pt}enc\ b\ n{\isacharparenright}{\kern0pt}\ {\isacharequal}{\kern0pt}\ enc{\isacharunderscore}{\kern0pt}len\ b\ n{\isachardoublequoteclose}\isanewline
%
\isadelimproof
\ \ %
\endisadelimproof
%
\isatagproof
\isacommand{by}\isamarkupfalse%
\ {\isacharparenleft}{\kern0pt}induction\ b\ n\ rule{\isacharcolon}{\kern0pt}\ enc{\isacharunderscore}{\kern0pt}len{\isachardot}{\kern0pt}induct{\isacharparenright}{\kern0pt}\ auto%
\endisatagproof
{\isafoldproof}%
%
\isadelimproof
\isanewline
%
\endisadelimproof
\isanewline
\isacommand{lemma}\isamarkupfalse%
\ length{\isacharunderscore}{\kern0pt}lenc{\isacharcolon}{\kern0pt}\isanewline
\ \ {\isachardoublequoteopen}length\ {\isacharparenleft}{\kern0pt}lenc\ k\ b\ n{\isacharparenright}{\kern0pt}\ {\isacharequal}{\kern0pt}\ k{\isachardoublequoteclose}\isanewline
%
\isadelimproof
\ \ %
\endisadelimproof
%
\isatagproof
\isacommand{by}\isamarkupfalse%
\ {\isacharparenleft}{\kern0pt}induction\ k\ arbitrary{\isacharcolon}{\kern0pt}\ n{\isacharparenright}{\kern0pt}\ auto%
\endisatagproof
{\isafoldproof}%
%
\isadelimproof
\isanewline
%
\endisadelimproof
\isanewline
\isacommand{lemma}\isamarkupfalse%
\ val{\isacharunderscore}{\kern0pt}correct{\isacharcolon}{\kern0pt}\isanewline
\ \ {\isachardoublequoteopen}valid\ b\ w\ {\isasymLongrightarrow}\ lenc\ {\isacharparenleft}{\kern0pt}length\ w{\isacharparenright}{\kern0pt}\ b\ {\isacharparenleft}{\kern0pt}val\ b\ w{\isacharparenright}{\kern0pt}\ {\isacharequal}{\kern0pt}\ w{\isachardoublequoteclose}\isanewline
%
\isadelimproof
\ \ %
\endisadelimproof
%
\isatagproof
\isacommand{by}\isamarkupfalse%
\ {\isacharparenleft}{\kern0pt}induction\ w{\isacharparenright}{\kern0pt}\ auto%
\endisatagproof
{\isafoldproof}%
%
\isadelimproof
\isanewline
%
\endisadelimproof
\isanewline
\isacommand{lemma}\isamarkupfalse%
\ val{\isacharunderscore}{\kern0pt}enc{\isacharcolon}{\kern0pt}\isanewline
\ \ {\isachardoublequoteopen}{\isadigit{2}}{\isasymle}b\ {\isasymLongrightarrow}\ val\ b\ {\isacharparenleft}{\kern0pt}enc\ b\ n{\isacharparenright}{\kern0pt}\ {\isacharequal}{\kern0pt}\ n{\isachardoublequoteclose}\isanewline
%
\isadelimproof
\ \ %
\endisadelimproof
%
\isatagproof
\isacommand{by}\isamarkupfalse%
\ {\isacharparenleft}{\kern0pt}induction\ b\ n\ rule{\isacharcolon}{\kern0pt}\ enc{\isachardot}{\kern0pt}induct{\isacharparenright}{\kern0pt}\ auto%
\endisatagproof
{\isafoldproof}%
%
\isadelimproof
\isanewline
%
\endisadelimproof
\isanewline
\isacommand{lemma}\isamarkupfalse%
\ val{\isacharunderscore}{\kern0pt}lenc{\isacharcolon}{\kern0pt}\isanewline
\ \ {\isachardoublequoteopen}val\ b\ {\isacharparenleft}{\kern0pt}lenc\ k\ b\ n{\isacharparenright}{\kern0pt}\ {\isacharequal}{\kern0pt}\ n\ mod\ b{\isacharcircum}{\kern0pt}k{\isachardoublequoteclose}\isanewline
%
\isadelimproof
\ \ %
\endisadelimproof
%
\isatagproof
\isacommand{apply}\isamarkupfalse%
\ {\isacharparenleft}{\kern0pt}induction\ k\ arbitrary{\isacharcolon}{\kern0pt}\ n{\isacharparenright}{\kern0pt}\isanewline
\ \ \isacommand{by}\isamarkupfalse%
\ {\isacharparenleft}{\kern0pt}auto\ simp\ add{\isacharcolon}{\kern0pt}\ mod{\isacharunderscore}{\kern0pt}mult{\isadigit{2}}{\isacharunderscore}{\kern0pt}eq{\isacharparenright}{\kern0pt}%
\endisatagproof
{\isafoldproof}%
%
\isadelimproof
\isanewline
%
\endisadelimproof
\isanewline
\isacommand{lemma}\isamarkupfalse%
\ valid{\isacharunderscore}{\kern0pt}enc{\isacharcolon}{\kern0pt}\isanewline
\ \ {\isachardoublequoteopen}{\isadigit{2}}{\isasymle}b\ {\isasymLongrightarrow}\ valid\ b\ {\isacharparenleft}{\kern0pt}enc\ b\ n{\isacharparenright}{\kern0pt}{\isachardoublequoteclose}\isanewline
%
\isadelimproof
\ \ %
\endisadelimproof
%
\isatagproof
\isacommand{by}\isamarkupfalse%
\ {\isacharparenleft}{\kern0pt}induction\ b\ n\ rule{\isacharcolon}{\kern0pt}\ enc{\isachardot}{\kern0pt}induct{\isacharparenright}{\kern0pt}\ auto%
\endisatagproof
{\isafoldproof}%
%
\isadelimproof
\isanewline
%
\endisadelimproof
\isanewline
\isacommand{lemma}\isamarkupfalse%
\ valid{\isacharunderscore}{\kern0pt}lenc{\isacharcolon}{\kern0pt}\isanewline
\ \ {\isachardoublequoteopen}{\isadigit{2}}{\isasymle}b\ {\isasymLongrightarrow}\ valid\ b\ {\isacharparenleft}{\kern0pt}lenc\ k\ b\ n{\isacharparenright}{\kern0pt}{\isachardoublequoteclose}\isanewline
%
\isadelimproof
\ \ %
\endisadelimproof
%
\isatagproof
\isacommand{by}\isamarkupfalse%
\ {\isacharparenleft}{\kern0pt}induction\ k\ arbitrary{\isacharcolon}{\kern0pt}\ n{\isacharparenright}{\kern0pt}\ auto%
\endisatagproof
{\isafoldproof}%
%
\isadelimproof
\isanewline
%
\endisadelimproof
\isanewline
\isacommand{lemma}\isamarkupfalse%
\ encodings{\isacharunderscore}{\kern0pt}agree{\isacharcolon}{\kern0pt}\isanewline
\ \ {\isachardoublequoteopen}{\isadigit{2}}{\isasymle}b\ {\isasymLongrightarrow}\ lenc\ {\isacharparenleft}{\kern0pt}enc{\isacharunderscore}{\kern0pt}len\ b\ n{\isacharparenright}{\kern0pt}\ b\ n\ {\isacharequal}{\kern0pt}\ enc\ b\ n{\isachardoublequoteclose}\isanewline
%
\isadelimproof
\ \ %
\endisadelimproof
%
\isatagproof
\isacommand{by}\isamarkupfalse%
\ {\isacharparenleft}{\kern0pt}metis\ length{\isacharunderscore}{\kern0pt}enc\ val{\isacharunderscore}{\kern0pt}correct\ val{\isacharunderscore}{\kern0pt}enc\ valid{\isacharunderscore}{\kern0pt}enc{\isacharparenright}{\kern0pt}%
\endisatagproof
{\isafoldproof}%
%
\isadelimproof
\isanewline
%
\endisadelimproof
\isanewline
\isacommand{lemma}\isamarkupfalse%
\ inj{\isacharunderscore}{\kern0pt}enc{\isacharcolon}{\kern0pt}\isanewline
\ \ {\isachardoublequoteopen}{\isadigit{2}}{\isasymle}b\ {\isasymLongrightarrow}\ inj\ {\isacharparenleft}{\kern0pt}enc\ b{\isacharparenright}{\kern0pt}{\isachardoublequoteclose}\isanewline
%
\isadelimproof
\ \ %
\endisadelimproof
%
\isatagproof
\isacommand{by}\isamarkupfalse%
\ {\isacharparenleft}{\kern0pt}metis\ val{\isacharunderscore}{\kern0pt}enc\ injI{\isacharparenright}{\kern0pt}%
\endisatagproof
{\isafoldproof}%
%
\isadelimproof
\isanewline
%
\endisadelimproof
\isanewline
\isacommand{lemma}\isamarkupfalse%
\ inj{\isacharunderscore}{\kern0pt}lenc{\isacharcolon}{\kern0pt}\isanewline
\ \ {\isachardoublequoteopen}inj{\isacharunderscore}{\kern0pt}on\ {\isacharparenleft}{\kern0pt}lenc\ k\ b{\isacharparenright}{\kern0pt}\ {\isacharbraceleft}{\kern0pt}{\isachardot}{\kern0pt}{\isachardot}{\kern0pt}{\isacharless}{\kern0pt}b{\isacharcircum}{\kern0pt}k{\isacharbraceright}{\kern0pt}{\isachardoublequoteclose}\isanewline
%
\isadelimproof
%
\endisadelimproof
%
\isatagproof
\isacommand{proof}\isamarkupfalse%
\ {\isacharparenleft}{\kern0pt}rule\ inj{\isacharunderscore}{\kern0pt}on{\isacharunderscore}{\kern0pt}inverseI{\isacharparenright}{\kern0pt}\isanewline
\ \ \isacommand{fix}\isamarkupfalse%
\ n\ {\isacharcolon}{\kern0pt}{\isacharcolon}{\kern0pt}\ nat\isanewline
\ \ \isacommand{assume}\isamarkupfalse%
\ {\isachardoublequoteopen}n\ {\isasymin}\ {\isacharbraceleft}{\kern0pt}{\isachardot}{\kern0pt}{\isachardot}{\kern0pt}{\isacharless}{\kern0pt}b{\isacharcircum}{\kern0pt}k{\isacharbraceright}{\kern0pt}{\isachardoublequoteclose}\isanewline
\ \ \isacommand{thus}\isamarkupfalse%
\ {\isachardoublequoteopen}val\ b\ {\isacharparenleft}{\kern0pt}lenc\ k\ b\ n{\isacharparenright}{\kern0pt}\ {\isacharequal}{\kern0pt}\ n{\isachardoublequoteclose}\ \isacommand{by}\isamarkupfalse%
\ {\isacharparenleft}{\kern0pt}simp\ add{\isacharcolon}{\kern0pt}\ val{\isacharunderscore}{\kern0pt}lenc{\isacharparenright}{\kern0pt}\isanewline
\isacommand{qed}\isamarkupfalse%
%
\endisatagproof
{\isafoldproof}%
%
\isadelimproof
\isanewline
%
\endisadelimproof
\isanewline
\isacommand{lemma}\isamarkupfalse%
\ normal{\isacharunderscore}{\kern0pt}enc{\isacharcolon}{\kern0pt}\isanewline
\ \ {\isachardoublequoteopen}{\isadigit{2}}{\isasymle}b\ {\isasymLongrightarrow}\ normal\ b\ {\isacharparenleft}{\kern0pt}enc\ b\ n{\isacharparenright}{\kern0pt}{\isachardoublequoteclose}\isanewline
%
\isadelimproof
\ \ %
\endisadelimproof
%
\isatagproof
\isacommand{by}\isamarkupfalse%
\ {\isacharparenleft}{\kern0pt}simp\ add{\isacharcolon}{\kern0pt}\ length{\isacharunderscore}{\kern0pt}enc\ normal{\isacharunderscore}{\kern0pt}def\ val{\isacharunderscore}{\kern0pt}enc{\isacharparenright}{\kern0pt}%
\endisatagproof
{\isafoldproof}%
%
\isadelimproof
\isanewline
%
\endisadelimproof
\isanewline
\isacommand{lemma}\isamarkupfalse%
\ normal{\isacharunderscore}{\kern0pt}eq{\isacharcolon}{\kern0pt}\isanewline
\ \ {\isachardoublequoteopen}{\isasymlbrakk}valid\ b\ v{\isacharsemicolon}{\kern0pt}\ valid\ b\ w{\isacharsemicolon}{\kern0pt}\ normal\ b\ v{\isacharsemicolon}{\kern0pt}\ normal\ b\ w{\isacharsemicolon}{\kern0pt}\ val\ b\ v\ {\isacharequal}{\kern0pt}\ val\ b\ w{\isasymrbrakk}\ {\isasymLongrightarrow}\ v\ {\isacharequal}{\kern0pt}\ w{\isachardoublequoteclose}\isanewline
%
\isadelimproof
\ \ %
\endisadelimproof
%
\isatagproof
\isacommand{by}\isamarkupfalse%
\ {\isacharparenleft}{\kern0pt}metis\ normal{\isacharunderscore}{\kern0pt}def\ val{\isacharunderscore}{\kern0pt}correct{\isacharparenright}{\kern0pt}%
\endisatagproof
{\isafoldproof}%
%
\isadelimproof
\isanewline
%
\endisadelimproof
\isanewline
\isacommand{lemma}\isamarkupfalse%
\ inj{\isacharunderscore}{\kern0pt}val{\isacharcolon}{\kern0pt}\isanewline
\ \ {\isachardoublequoteopen}inj{\isacharunderscore}{\kern0pt}on\ {\isacharparenleft}{\kern0pt}val\ b{\isacharparenright}{\kern0pt}\ {\isacharbraceleft}{\kern0pt}w{\isachardot}{\kern0pt}\ valid\ b\ w\ {\isasymand}\ normal\ b\ w{\isacharbraceright}{\kern0pt}{\isachardoublequoteclose}\isanewline
%
\isadelimproof
%
\endisadelimproof
%
\isatagproof
\isacommand{proof}\isamarkupfalse%
\ {\isacharparenleft}{\kern0pt}rule\ inj{\isacharunderscore}{\kern0pt}onI{\isacharparenright}{\kern0pt}\isanewline
\ \ \isacommand{fix}\isamarkupfalse%
\ u\ v\ {\isacharcolon}{\kern0pt}{\isacharcolon}{\kern0pt}\ word\isanewline
\ \ \isacommand{assume}\isamarkupfalse%
\ {\isadigit{1}}{\isacharcolon}{\kern0pt}\ {\isachardoublequoteopen}val\ b\ u\ {\isacharequal}{\kern0pt}\ val\ b\ v{\isachardoublequoteclose}\isanewline
\ \ \isacommand{assume}\isamarkupfalse%
\ {\isachardoublequoteopen}u\ {\isasymin}\ {\isacharbraceleft}{\kern0pt}w{\isachardot}{\kern0pt}\ valid\ b\ w\ {\isasymand}\ normal\ b\ w{\isacharbraceright}{\kern0pt}{\isachardoublequoteclose}\isanewline
\ \ \ \ \ \isakeyword{and}\ {\isachardoublequoteopen}v\ {\isasymin}\ {\isacharbraceleft}{\kern0pt}w{\isachardot}{\kern0pt}\ valid\ b\ w\ {\isasymand}\ normal\ b\ w{\isacharbraceright}{\kern0pt}{\isachardoublequoteclose}\isanewline
\ \ \isacommand{hence}\isamarkupfalse%
\ {\isachardoublequoteopen}valid\ b\ u\ {\isasymand}\ normal\ b\ u\ {\isasymand}\ valid\ b\ v\ {\isasymand}\ normal\ b\ v{\isachardoublequoteclose}\ \isacommand{by}\isamarkupfalse%
\ blast\isanewline
\ \ \isacommand{with}\isamarkupfalse%
\ {\isachardoublequoteopen}{\isadigit{1}}{\isachardoublequoteclose}\ \isacommand{show}\isamarkupfalse%
\ {\isachardoublequoteopen}u\ {\isacharequal}{\kern0pt}\ v{\isachardoublequoteclose}\ \isacommand{using}\isamarkupfalse%
\ normal{\isacharunderscore}{\kern0pt}eq\ \isacommand{by}\isamarkupfalse%
\ blast\isanewline
\isacommand{qed}\isamarkupfalse%
%
\endisatagproof
{\isafoldproof}%
%
\isadelimproof
\isanewline
%
\endisadelimproof
\isanewline
\isacommand{lemma}\isamarkupfalse%
\ enc{\isacharunderscore}{\kern0pt}val{\isacharcolon}{\kern0pt}\isanewline
\ \ {\isachardoublequoteopen}{\isasymlbrakk}valid\ b\ w{\isacharsemicolon}{\kern0pt}\ normal\ b\ w{\isasymrbrakk}\ {\isasymLongrightarrow}\ enc\ b\ {\isacharparenleft}{\kern0pt}val\ b\ w{\isacharparenright}{\kern0pt}\ {\isacharequal}{\kern0pt}\ w{\isachardoublequoteclose}\isanewline
%
\isadelimproof
\ \ %
\endisadelimproof
%
\isatagproof
\isacommand{by}\isamarkupfalse%
\ {\isacharparenleft}{\kern0pt}metis\ encodings{\isacharunderscore}{\kern0pt}agree\ normal{\isacharunderscore}{\kern0pt}def\ val{\isacharunderscore}{\kern0pt}correct\ valid{\isacharunderscore}{\kern0pt}base{\isacharparenright}{\kern0pt}%
\endisatagproof
{\isafoldproof}%
%
\isadelimproof
\isanewline
%
\endisadelimproof
\isanewline
\isacommand{lemma}\isamarkupfalse%
\ range{\isacharunderscore}{\kern0pt}enc{\isacharcolon}{\kern0pt}\isanewline
\ \ {\isachardoublequoteopen}{\isadigit{2}}{\isasymle}b\ {\isasymLongrightarrow}\ range\ {\isacharparenleft}{\kern0pt}enc\ b{\isacharparenright}{\kern0pt}\ {\isacharequal}{\kern0pt}\ {\isacharbraceleft}{\kern0pt}w{\isachardot}{\kern0pt}\ valid\ b\ w\ {\isasymand}\ normal\ b\ w{\isacharbraceright}{\kern0pt}{\isachardoublequoteclose}\isanewline
%
\isadelimproof
%
\endisadelimproof
%
\isatagproof
\isacommand{proof}\isamarkupfalse%
\isanewline
\ \ \isacommand{show}\isamarkupfalse%
\ {\isachardoublequoteopen}{\isadigit{2}}{\isasymle}b\ {\isasymLongrightarrow}\ range\ {\isacharparenleft}{\kern0pt}enc\ b{\isacharparenright}{\kern0pt}\ {\isasymsubseteq}\ {\isacharbraceleft}{\kern0pt}w{\isachardot}{\kern0pt}\ valid\ b\ w\ {\isasymand}\ normal\ b\ w{\isacharbraceright}{\kern0pt}{\isachardoublequoteclose}\isanewline
\ \ \ \ \isacommand{by}\isamarkupfalse%
\ {\isacharparenleft}{\kern0pt}simp\ add{\isacharcolon}{\kern0pt}\ image{\isacharunderscore}{\kern0pt}subsetI\ normal{\isacharunderscore}{\kern0pt}enc\ valid{\isacharunderscore}{\kern0pt}enc{\isacharparenright}{\kern0pt}\isanewline
\isacommand{next}\isamarkupfalse%
\isanewline
\ \ \isacommand{assume}\isamarkupfalse%
\ {\isachardoublequoteopen}{\isadigit{2}}{\isasymle}b{\isachardoublequoteclose}\isanewline
\ \ \isacommand{show}\isamarkupfalse%
\ {\isachardoublequoteopen}{\isacharbraceleft}{\kern0pt}w{\isachardot}{\kern0pt}\ valid\ b\ w\ {\isasymand}\ normal\ b\ w{\isacharbraceright}{\kern0pt}\ {\isasymsubseteq}\ range\ {\isacharparenleft}{\kern0pt}enc\ b{\isacharparenright}{\kern0pt}{\isachardoublequoteclose}\isanewline
\ \ \isacommand{proof}\isamarkupfalse%
\isanewline
\ \ \ \ \isacommand{fix}\isamarkupfalse%
\ v\ {\isacharcolon}{\kern0pt}{\isacharcolon}{\kern0pt}\ word\isanewline
\ \ \ \ \isacommand{assume}\isamarkupfalse%
\ {\isachardoublequoteopen}v\ {\isasymin}\ {\isacharbraceleft}{\kern0pt}w{\isachardot}{\kern0pt}\ valid\ b\ w\ {\isasymand}\ normal\ b\ w{\isacharbraceright}{\kern0pt}{\isachardoublequoteclose}\isanewline
\ \ \ \ \isacommand{hence}\isamarkupfalse%
\ {\isachardoublequoteopen}valid\ b\ v\ {\isasymand}\ normal\ b\ v{\isachardoublequoteclose}\ \isacommand{by}\isamarkupfalse%
\ blast\isanewline
\ \ \ \ \isacommand{hence}\isamarkupfalse%
\ {\isachardoublequoteopen}enc\ b\ {\isacharparenleft}{\kern0pt}val\ b\ v{\isacharparenright}{\kern0pt}\ {\isacharequal}{\kern0pt}\ v{\isachardoublequoteclose}\ \isacommand{by}\isamarkupfalse%
\ {\isacharparenleft}{\kern0pt}simp\ add{\isacharcolon}{\kern0pt}\ enc{\isacharunderscore}{\kern0pt}val{\isacharparenright}{\kern0pt}\isanewline
\ \ \ \ \isacommand{thus}\isamarkupfalse%
\ {\isachardoublequoteopen}v\ {\isasymin}\ range\ {\isacharparenleft}{\kern0pt}enc\ b{\isacharparenright}{\kern0pt}{\isachardoublequoteclose}\ \isacommand{by}\isamarkupfalse%
\ {\isacharparenleft}{\kern0pt}metis\ rangeI{\isacharparenright}{\kern0pt}\isanewline
\ \ \isacommand{qed}\isamarkupfalse%
\isanewline
\isacommand{qed}\isamarkupfalse%
%
\endisatagproof
{\isafoldproof}%
%
\isadelimproof
\isanewline
%
\endisadelimproof
\isanewline
\isacommand{lemma}\isamarkupfalse%
\ range{\isacharunderscore}{\kern0pt}lenc{\isacharcolon}{\kern0pt}\isanewline
\ \ {\isachardoublequoteopen}{\isadigit{2}}{\isasymle}b\ {\isasymLongrightarrow}\ lenc\ k\ b\ {\isacharbackquote}{\kern0pt}\ {\isacharbraceleft}{\kern0pt}{\isachardot}{\kern0pt}{\isachardot}{\kern0pt}{\isacharless}{\kern0pt}b\ {\isacharcircum}{\kern0pt}\ k{\isacharbraceright}{\kern0pt}\ {\isacharequal}{\kern0pt}\ {\isacharbraceleft}{\kern0pt}w{\isachardot}{\kern0pt}\ valid\ b\ w\ {\isasymand}\ length\ w\ {\isacharequal}{\kern0pt}\ k{\isacharbraceright}{\kern0pt}{\isachardoublequoteclose}\isanewline
%
\isadelimproof
%
\endisadelimproof
%
\isatagproof
\isacommand{proof}\isamarkupfalse%
\isanewline
\ \ \isacommand{show}\isamarkupfalse%
\ {\isachardoublequoteopen}{\isadigit{2}}\ {\isasymle}\ b\ {\isasymLongrightarrow}\ lenc\ k\ b\ {\isacharbackquote}{\kern0pt}\ {\isacharbraceleft}{\kern0pt}{\isachardot}{\kern0pt}{\isachardot}{\kern0pt}{\isacharless}{\kern0pt}b\ {\isacharcircum}{\kern0pt}\ k{\isacharbraceright}{\kern0pt}\ {\isasymsubseteq}\ {\isacharbraceleft}{\kern0pt}w{\isachardot}{\kern0pt}\ valid\ b\ w\ {\isasymand}\ length\ w\ {\isacharequal}{\kern0pt}\ k{\isacharbraceright}{\kern0pt}{\isachardoublequoteclose}\isanewline
\ \ \ \ \isacommand{by}\isamarkupfalse%
\ {\isacharparenleft}{\kern0pt}simp\ add{\isacharcolon}{\kern0pt}\ image{\isacharunderscore}{\kern0pt}subsetI\ length{\isacharunderscore}{\kern0pt}lenc\ valid{\isacharunderscore}{\kern0pt}lenc{\isacharparenright}{\kern0pt}\isanewline
\isacommand{next}\isamarkupfalse%
\isanewline
\ \ \isacommand{assume}\isamarkupfalse%
\ {\isachardoublequoteopen}{\isadigit{2}}{\isasymle}b{\isachardoublequoteclose}\isanewline
\ \ \isacommand{show}\isamarkupfalse%
\ {\isachardoublequoteopen}{\isacharbraceleft}{\kern0pt}w{\isachardot}{\kern0pt}\ valid\ b\ w\ {\isasymand}\ length\ w\ {\isacharequal}{\kern0pt}\ k{\isacharbraceright}{\kern0pt}\ {\isasymsubseteq}\ lenc\ k\ b\ {\isacharbackquote}{\kern0pt}\ {\isacharbraceleft}{\kern0pt}{\isachardot}{\kern0pt}{\isachardot}{\kern0pt}{\isacharless}{\kern0pt}b\ {\isacharcircum}{\kern0pt}\ k{\isacharbraceright}{\kern0pt}{\isachardoublequoteclose}\isanewline
\ \ \isacommand{proof}\isamarkupfalse%
\isanewline
\ \ \ \ \isacommand{fix}\isamarkupfalse%
\ v\ {\isacharcolon}{\kern0pt}{\isacharcolon}{\kern0pt}\ word\isanewline
\ \ \ \ \isacommand{let}\isamarkupfalse%
\ {\isacharquery}{\kern0pt}v\ {\isacharequal}{\kern0pt}\ {\isachardoublequoteopen}val\ b\ v{\isachardoublequoteclose}\isanewline
\ \ \ \ \isacommand{assume}\isamarkupfalse%
\ {\isachardoublequoteopen}v\ {\isasymin}\ {\isacharbraceleft}{\kern0pt}w{\isachardot}{\kern0pt}\ valid\ b\ w\ {\isasymand}\ length\ w\ {\isacharequal}{\kern0pt}\ k{\isacharbraceright}{\kern0pt}{\isachardoublequoteclose}\isanewline
\ \ \ \ \isacommand{hence}\isamarkupfalse%
\ {\isadigit{1}}{\isacharcolon}{\kern0pt}\ {\isachardoublequoteopen}valid\ b\ v\ {\isasymand}\ length\ v\ {\isacharequal}{\kern0pt}\ k{\isachardoublequoteclose}\ \isacommand{by}\isamarkupfalse%
\ blast\isanewline
\ \ \ \ \isacommand{hence}\isamarkupfalse%
\ {\isachardoublequoteopen}{\isacharquery}{\kern0pt}v\ {\isacharless}{\kern0pt}\ b{\isacharcircum}{\kern0pt}k{\isachardoublequoteclose}\ \isacommand{using}\isamarkupfalse%
\ val{\isacharunderscore}{\kern0pt}bound\ \isacommand{by}\isamarkupfalse%
\ blast\isanewline
\ \ \ \ \isacommand{hence}\isamarkupfalse%
\ {\isachardoublequoteopen}{\isacharquery}{\kern0pt}v\ {\isasymin}\ {\isacharbraceleft}{\kern0pt}{\isachardot}{\kern0pt}{\isachardot}{\kern0pt}{\isacharless}{\kern0pt}b{\isacharcircum}{\kern0pt}k{\isacharbraceright}{\kern0pt}{\isachardoublequoteclose}\ \isacommand{by}\isamarkupfalse%
\ blast\isanewline
\ \ \ \ \isacommand{from}\isamarkupfalse%
\ {\isachardoublequoteopen}{\isadigit{1}}{\isachardoublequoteclose}\ \isacommand{have}\isamarkupfalse%
\ {\isachardoublequoteopen}lenc\ k\ b\ {\isacharquery}{\kern0pt}v\ {\isacharequal}{\kern0pt}\ v{\isachardoublequoteclose}\ \isacommand{using}\isamarkupfalse%
\ val{\isacharunderscore}{\kern0pt}correct\ \isacommand{by}\isamarkupfalse%
\ blast\isanewline
\ \ \ \ \isacommand{thus}\isamarkupfalse%
\ {\isachardoublequoteopen}v\ {\isasymin}\ lenc\ k\ b\ {\isacharbackquote}{\kern0pt}\ {\isacharbraceleft}{\kern0pt}{\isachardot}{\kern0pt}{\isachardot}{\kern0pt}{\isacharless}{\kern0pt}b\ {\isacharcircum}{\kern0pt}\ k{\isacharbraceright}{\kern0pt}{\isachardoublequoteclose}\ \isacommand{by}\isamarkupfalse%
\ {\isacharparenleft}{\kern0pt}metis\ {\isacartoucheopen}{\isacharquery}{\kern0pt}v\ {\isasymin}\ {\isacharbraceleft}{\kern0pt}{\isachardot}{\kern0pt}{\isachardot}{\kern0pt}{\isacharless}{\kern0pt}b{\isacharcircum}{\kern0pt}k{\isacharbraceright}{\kern0pt}{\isacartoucheclose}\ image{\isacharunderscore}{\kern0pt}eqI{\isacharparenright}{\kern0pt}\isanewline
\ \ \isacommand{qed}\isamarkupfalse%
\isanewline
\isacommand{qed}\isamarkupfalse%
%
\endisatagproof
{\isafoldproof}%
%
\isadelimproof
\isanewline
%
\endisadelimproof
\isanewline
\isacommand{theorem}\isamarkupfalse%
\ enc{\isacharunderscore}{\kern0pt}correct{\isacharcolon}{\kern0pt}\isanewline
\ \ {\isachardoublequoteopen}{\isadigit{2}}{\isasymle}b\ {\isasymLongrightarrow}\ bij{\isacharunderscore}{\kern0pt}betw\ {\isacharparenleft}{\kern0pt}enc\ b{\isacharparenright}{\kern0pt}\ UNIV\ {\isacharbraceleft}{\kern0pt}w{\isachardot}{\kern0pt}\ valid\ b\ w\ {\isasymand}\ normal\ b\ w{\isacharbraceright}{\kern0pt}{\isachardoublequoteclose}\isanewline
%
\isadelimproof
\ \ %
\endisadelimproof
%
\isatagproof
\isacommand{by}\isamarkupfalse%
\ {\isacharparenleft}{\kern0pt}simp\ add{\isacharcolon}{\kern0pt}\ bij{\isacharunderscore}{\kern0pt}betw{\isacharunderscore}{\kern0pt}def\ inj{\isacharunderscore}{\kern0pt}enc\ range{\isacharunderscore}{\kern0pt}enc{\isacharparenright}{\kern0pt}%
\endisatagproof
{\isafoldproof}%
%
\isadelimproof
%
\endisadelimproof
%
\begin{isamarkuptext}%
Given a valid base $b$ and length $k$, we encode exactly the first $b^k$ numbers.%
\end{isamarkuptext}\isamarkuptrue%
\isacommand{theorem}\isamarkupfalse%
\ lenc{\isacharunderscore}{\kern0pt}correct{\isacharcolon}{\kern0pt}\isanewline
\ \ {\isachardoublequoteopen}{\isadigit{2}}{\isasymle}b\ {\isasymLongrightarrow}\ bij{\isacharunderscore}{\kern0pt}betw\ {\isacharparenleft}{\kern0pt}lenc\ k\ b{\isacharparenright}{\kern0pt}\ {\isacharbraceleft}{\kern0pt}{\isachardot}{\kern0pt}{\isachardot}{\kern0pt}{\isacharless}{\kern0pt}b{\isacharcircum}{\kern0pt}k{\isacharbraceright}{\kern0pt}\ {\isacharbraceleft}{\kern0pt}w{\isachardot}{\kern0pt}\ valid\ b\ w\ {\isasymand}\ length\ w\ {\isacharequal}{\kern0pt}\ k{\isacharbraceright}{\kern0pt}{\isachardoublequoteclose}\isanewline
%
\isadelimproof
\ \ %
\endisadelimproof
%
\isatagproof
\isacommand{by}\isamarkupfalse%
\ {\isacharparenleft}{\kern0pt}simp\ add{\isacharcolon}{\kern0pt}\ bij{\isacharunderscore}{\kern0pt}betw{\isacharunderscore}{\kern0pt}def\ inj{\isacharunderscore}{\kern0pt}lenc\ range{\isacharunderscore}{\kern0pt}lenc{\isacharparenright}{\kern0pt}%
\endisatagproof
{\isafoldproof}%
%
\isadelimproof
%
\endisadelimproof
%
\isadelimdocument
%
\endisadelimdocument
%
\isatagdocument
%
\isamarkupsubsection{Circular Increment Operation%
}
\isamarkuptrue%
%
\endisatagdocument
{\isafolddocument}%
%
\isadelimdocument
%
\endisadelimdocument
%
\begin{isamarkuptext}%
It is beneficial for our purpose to have an increment operation on
  words of fixed length that wraps around.
Mathematically, this corresponds to adding 1 in the additive group
  of the factor ring of the integers modulo ($b^k$).
Correctness is proven in terms of previously verified operations.%
\end{isamarkuptext}\isamarkuptrue%
\isacommand{fun}\isamarkupfalse%
\ inc\ {\isacharcolon}{\kern0pt}{\isacharcolon}{\kern0pt}\ {\isachardoublequoteopen}nat\ {\isasymRightarrow}\ word\ {\isasymRightarrow}\ word{\isachardoublequoteclose}\ \isakeyword{where}\isanewline
\ \ {\isachardoublequoteopen}inc\ {\isacharunderscore}{\kern0pt}\ {\isacharbrackleft}{\kern0pt}{\isacharbrackright}{\kern0pt}\ {\isacharequal}{\kern0pt}\ {\isacharbrackleft}{\kern0pt}{\isacharbrackright}{\kern0pt}{\isachardoublequoteclose}\isanewline
{\isacharbar}{\kern0pt}\ {\isachardoublequoteopen}inc\ b\ {\isacharparenleft}{\kern0pt}a{\isacharhash}{\kern0pt}w{\isacharparenright}{\kern0pt}\ {\isacharequal}{\kern0pt}\ Suc\ a\ mod\ b{\isacharhash}{\kern0pt}{\isacharparenleft}{\kern0pt}if\ Suc\ a\ {\isasymnoteq}\ b\ then\ w\ else\ inc\ b\ w{\isacharparenright}{\kern0pt}{\isachardoublequoteclose}\isanewline
\isanewline
\isacommand{lemma}\isamarkupfalse%
\ length{\isacharunderscore}{\kern0pt}inc{\isacharcolon}{\kern0pt}\isanewline
\ \ {\isachardoublequoteopen}length\ {\isacharparenleft}{\kern0pt}inc\ b\ w{\isacharparenright}{\kern0pt}\ {\isacharequal}{\kern0pt}\ length\ w{\isachardoublequoteclose}\isanewline
%
\isadelimproof
\ \ %
\endisadelimproof
%
\isatagproof
\isacommand{by}\isamarkupfalse%
\ {\isacharparenleft}{\kern0pt}induction\ w{\isacharparenright}{\kern0pt}\ auto%
\endisatagproof
{\isafoldproof}%
%
\isadelimproof
\isanewline
%
\endisadelimproof
\isanewline
\isacommand{lemma}\isamarkupfalse%
\ valid{\isacharunderscore}{\kern0pt}inc{\isacharcolon}{\kern0pt}\isanewline
\ \ {\isachardoublequoteopen}valid\ b\ w\ {\isasymLongrightarrow}\ valid\ b\ {\isacharparenleft}{\kern0pt}inc\ b\ w{\isacharparenright}{\kern0pt}{\isachardoublequoteclose}\isanewline
%
\isadelimproof
\ \ %
\endisadelimproof
%
\isatagproof
\isacommand{by}\isamarkupfalse%
\ {\isacharparenleft}{\kern0pt}induction\ w{\isacharparenright}{\kern0pt}\ auto%
\endisatagproof
{\isafoldproof}%
%
\isadelimproof
%
\endisadelimproof
%
\begin{isamarkuptext}%
Note that the following fact shows that we do not only have an encoding
  in the sense that it is a bijection but we also preserve a certain structure,
  that is necessary for the purpose of reasoning about Gray codes.%
\end{isamarkuptext}\isamarkuptrue%
\isacommand{theorem}\isamarkupfalse%
\ val{\isacharunderscore}{\kern0pt}inc{\isacharcolon}{\kern0pt}\isanewline
\ \ {\isachardoublequoteopen}valid\ b\ w\ {\isasymLongrightarrow}\ val\ b\ {\isacharparenleft}{\kern0pt}inc\ b\ w{\isacharparenright}{\kern0pt}\ {\isacharequal}{\kern0pt}\ Suc\ {\isacharparenleft}{\kern0pt}val\ b\ w{\isacharparenright}{\kern0pt}\ mod\ b{\isacharcircum}{\kern0pt}length{\isacharparenleft}{\kern0pt}w{\isacharparenright}{\kern0pt}{\isachardoublequoteclose}\isanewline
%
\isadelimproof
%
\endisadelimproof
%
\isatagproof
\isacommand{proof}\isamarkupfalse%
\ {\isacharparenleft}{\kern0pt}induction\ w{\isacharparenright}{\kern0pt}\isanewline
\ \ \isacommand{case}\isamarkupfalse%
\ Nil\ \isacommand{thus}\isamarkupfalse%
\ {\isacharquery}{\kern0pt}case\ \isacommand{by}\isamarkupfalse%
\ simp\isanewline
\isacommand{next}\isamarkupfalse%
\isanewline
\ \ \isacommand{case}\isamarkupfalse%
\ {\isacharparenleft}{\kern0pt}Cons\ a\ w{\isacharparenright}{\kern0pt}\isanewline
\ \ \isacommand{hence}\isamarkupfalse%
\ IH{\isacharcolon}{\kern0pt}\ {\isachardoublequoteopen}val\ b\ {\isacharparenleft}{\kern0pt}inc\ b\ w{\isacharparenright}{\kern0pt}\ {\isacharequal}{\kern0pt}\ Suc{\isacharparenleft}{\kern0pt}val\ b\ w{\isacharparenright}{\kern0pt}\ mod\ b{\isacharcircum}{\kern0pt}length{\isacharparenleft}{\kern0pt}w{\isacharparenright}{\kern0pt}{\isachardoublequoteclose}\ \isacommand{by}\isamarkupfalse%
\ simp\isanewline
\ \ \isacommand{show}\isamarkupfalse%
\ {\isacharquery}{\kern0pt}case\isanewline
\ \ \isacommand{proof}\isamarkupfalse%
\ cases\isanewline
\ \ \ \ \isacommand{assume}\isamarkupfalse%
\ {\isadigit{1}}{\isacharcolon}{\kern0pt}\ {\isachardoublequoteopen}Suc\ a\ {\isacharequal}{\kern0pt}\ b{\isachardoublequoteclose}\isanewline
\ \ \ \ \isacommand{hence}\isamarkupfalse%
\ {\isachardoublequoteopen}val\ b\ {\isacharparenleft}{\kern0pt}inc\ b\ {\isacharparenleft}{\kern0pt}a{\isacharhash}{\kern0pt}w{\isacharparenright}{\kern0pt}{\isacharparenright}{\kern0pt}\ {\isacharequal}{\kern0pt}\ b{\isacharasterisk}{\kern0pt}val\ b\ {\isacharparenleft}{\kern0pt}inc\ b\ w{\isacharparenright}{\kern0pt}{\isachardoublequoteclose}\ \isacommand{by}\isamarkupfalse%
\ simp\isanewline
\ \ \ \ \isacommand{also}\isamarkupfalse%
\ \isacommand{have}\isamarkupfalse%
\ {\isachardoublequoteopen}{\isachardot}{\kern0pt}{\isachardot}{\kern0pt}{\isachardot}{\kern0pt}\ {\isacharequal}{\kern0pt}\ b{\isacharasterisk}{\kern0pt}{\isacharparenleft}{\kern0pt}Suc{\isacharparenleft}{\kern0pt}val\ b\ w{\isacharparenright}{\kern0pt}\ mod\ b{\isacharcircum}{\kern0pt}length\ w{\isacharparenright}{\kern0pt}{\isachardoublequoteclose}\ \isacommand{using}\isamarkupfalse%
\ IH\ \isacommand{by}\isamarkupfalse%
\ simp\isanewline
\ \ \ \ \isacommand{also}\isamarkupfalse%
\ \isacommand{have}\isamarkupfalse%
\ {\isachardoublequoteopen}{\isachardot}{\kern0pt}{\isachardot}{\kern0pt}{\isachardot}{\kern0pt}\ {\isacharequal}{\kern0pt}\ b{\isacharasterisk}{\kern0pt}Suc{\isacharparenleft}{\kern0pt}val\ b\ w{\isacharparenright}{\kern0pt}\ mod\ {\isacharparenleft}{\kern0pt}b{\isacharasterisk}{\kern0pt}b{\isacharcircum}{\kern0pt}length\ w{\isacharparenright}{\kern0pt}{\isachardoublequoteclose}\ \isacommand{using}\isamarkupfalse%
\ mult{\isacharunderscore}{\kern0pt}mod{\isacharunderscore}{\kern0pt}right\ \isacommand{by}\isamarkupfalse%
\ blast\isanewline
\ \ \ \ \isacommand{also}\isamarkupfalse%
\ \isacommand{have}\isamarkupfalse%
\ {\isachardoublequoteopen}{\isachardot}{\kern0pt}{\isachardot}{\kern0pt}{\isachardot}{\kern0pt}\ {\isacharequal}{\kern0pt}\ {\isacharparenleft}{\kern0pt}Suc\ a\ {\isacharplus}{\kern0pt}\ b{\isacharasterisk}{\kern0pt}val\ b\ w{\isacharparenright}{\kern0pt}\ mod\ {\isacharparenleft}{\kern0pt}b{\isacharcircum}{\kern0pt}length{\isacharparenleft}{\kern0pt}a{\isacharhash}{\kern0pt}w{\isacharparenright}{\kern0pt}{\isacharparenright}{\kern0pt}{\isachardoublequoteclose}\ \isacommand{by}\isamarkupfalse%
\ {\isacharparenleft}{\kern0pt}simp\ add{\isacharcolon}{\kern0pt}\ {\isachardoublequoteopen}{\isadigit{1}}{\isachardoublequoteclose}{\isacharparenright}{\kern0pt}\isanewline
\ \ \ \ \isacommand{also}\isamarkupfalse%
\ \isacommand{have}\isamarkupfalse%
\ {\isachardoublequoteopen}{\isachardot}{\kern0pt}{\isachardot}{\kern0pt}{\isachardot}{\kern0pt}\ {\isacharequal}{\kern0pt}\ Suc{\isacharparenleft}{\kern0pt}val\ b\ {\isacharparenleft}{\kern0pt}a\ {\isacharhash}{\kern0pt}\ w{\isacharparenright}{\kern0pt}{\isacharparenright}{\kern0pt}\ mod\ {\isacharparenleft}{\kern0pt}b{\isacharcircum}{\kern0pt}length{\isacharparenleft}{\kern0pt}a{\isacharhash}{\kern0pt}w{\isacharparenright}{\kern0pt}{\isacharparenright}{\kern0pt}{\isachardoublequoteclose}\ \isacommand{by}\isamarkupfalse%
\ simp\isanewline
\ \ \ \ \isacommand{finally}\isamarkupfalse%
\ \isacommand{show}\isamarkupfalse%
\ {\isacharquery}{\kern0pt}thesis\ \isacommand{by}\isamarkupfalse%
\ blast\isanewline
\ \ \isacommand{next}\isamarkupfalse%
\isanewline
\ \ \ \ \isacommand{let}\isamarkupfalse%
\ {\isacharquery}{\kern0pt}v\ {\isacharequal}{\kern0pt}\ {\isachardoublequoteopen}Suc\ a\ {\isacharplus}{\kern0pt}\ b{\isacharasterisk}{\kern0pt}val\ b\ w{\isachardoublequoteclose}\isanewline
\ \ \ \ \isacommand{assume}\isamarkupfalse%
\ {\isadigit{2}}{\isacharcolon}{\kern0pt}\ {\isachardoublequoteopen}Suc\ a\ {\isasymnoteq}\ b{\isachardoublequoteclose}\isanewline
\ \ \ \ \isacommand{with}\isamarkupfalse%
\ Cons{\isachardot}{\kern0pt}prems\ \isacommand{have}\isamarkupfalse%
\ {\isachardoublequoteopen}valid\ b\ {\isacharparenleft}{\kern0pt}inc\ b\ {\isacharparenleft}{\kern0pt}a{\isacharhash}{\kern0pt}w{\isacharparenright}{\kern0pt}{\isacharparenright}{\kern0pt}{\isachardoublequoteclose}\ \isacommand{by}\isamarkupfalse%
\ simp\isanewline
\ \ \ \ \isacommand{hence}\isamarkupfalse%
\ {\isachardoublequoteopen}val\ b\ {\isacharparenleft}{\kern0pt}inc\ b\ {\isacharparenleft}{\kern0pt}a{\isacharhash}{\kern0pt}w{\isacharparenright}{\kern0pt}{\isacharparenright}{\kern0pt}\ {\isacharless}{\kern0pt}\ b{\isacharcircum}{\kern0pt}length{\isacharparenleft}{\kern0pt}inc\ b\ {\isacharparenleft}{\kern0pt}a{\isacharhash}{\kern0pt}w{\isacharparenright}{\kern0pt}{\isacharparenright}{\kern0pt}{\isachardoublequoteclose}\ \isacommand{using}\isamarkupfalse%
\ val{\isacharunderscore}{\kern0pt}bound\ \isacommand{by}\isamarkupfalse%
\ blast\isanewline
\ \ \ \ \isacommand{hence}\isamarkupfalse%
\ {\isachardoublequoteopen}val\ b\ {\isacharparenleft}{\kern0pt}inc\ b\ {\isacharparenleft}{\kern0pt}a{\isacharhash}{\kern0pt}w{\isacharparenright}{\kern0pt}{\isacharparenright}{\kern0pt}\ {\isacharless}{\kern0pt}\ b{\isacharcircum}{\kern0pt}length{\isacharparenleft}{\kern0pt}a{\isacharhash}{\kern0pt}w{\isacharparenright}{\kern0pt}{\isachardoublequoteclose}\ \isacommand{using}\isamarkupfalse%
\ length{\isacharunderscore}{\kern0pt}inc\ \isacommand{by}\isamarkupfalse%
\ metis\isanewline
\ \ \ \ \isacommand{hence}\isamarkupfalse%
\ {\isachardoublequoteopen}{\isacharquery}{\kern0pt}v\ {\isacharless}{\kern0pt}\ b{\isacharcircum}{\kern0pt}length{\isacharparenleft}{\kern0pt}a{\isacharhash}{\kern0pt}w{\isacharparenright}{\kern0pt}{\isachardoublequoteclose}\ \isacommand{using}\isamarkupfalse%
\ {\isachardoublequoteopen}{\isadigit{2}}{\isachardoublequoteclose}\ Cons{\isachardot}{\kern0pt}prems\ \isacommand{by}\isamarkupfalse%
\ simp\isanewline
\ \ \ \ \isacommand{hence}\isamarkupfalse%
\ {\isachardoublequoteopen}{\isacharquery}{\kern0pt}v\ {\isacharequal}{\kern0pt}\ {\isacharquery}{\kern0pt}v\ mod\ b{\isacharcircum}{\kern0pt}length{\isacharparenleft}{\kern0pt}a{\isacharhash}{\kern0pt}w{\isacharparenright}{\kern0pt}{\isachardoublequoteclose}\ \isacommand{by}\isamarkupfalse%
\ simp\isanewline
\ \ \ \ \isacommand{thus}\isamarkupfalse%
\ {\isacharquery}{\kern0pt}thesis\ \isacommand{using}\isamarkupfalse%
\ {\isachardoublequoteopen}{\isadigit{2}}{\isachardoublequoteclose}\ Cons{\isachardot}{\kern0pt}prems\ \isacommand{by}\isamarkupfalse%
\ auto\isanewline
\ \ \isacommand{qed}\isamarkupfalse%
\isanewline
\isacommand{qed}\isamarkupfalse%
%
\endisatagproof
{\isafoldproof}%
%
\isadelimproof
\isanewline
%
\endisadelimproof
\isanewline
\isacommand{lemma}\isamarkupfalse%
\ inc{\isacharunderscore}{\kern0pt}correct{\isacharcolon}{\kern0pt}\isanewline
\ \ {\isachardoublequoteopen}inc\ b\ {\isacharparenleft}{\kern0pt}lenc\ k\ b\ n{\isacharparenright}{\kern0pt}\ {\isacharequal}{\kern0pt}\ lenc\ k\ b\ {\isacharparenleft}{\kern0pt}Suc\ n{\isacharparenright}{\kern0pt}{\isachardoublequoteclose}\isanewline
%
\isadelimproof
\ \ %
\endisadelimproof
%
\isatagproof
\isacommand{apply}\isamarkupfalse%
\ {\isacharparenleft}{\kern0pt}induction\ k\ arbitrary{\isacharcolon}{\kern0pt}\ n{\isacharparenright}{\kern0pt}\isanewline
\ \ \isacommand{by}\isamarkupfalse%
\ {\isacharparenleft}{\kern0pt}auto\ simp\ add{\isacharcolon}{\kern0pt}\ div{\isacharunderscore}{\kern0pt}Suc\ mod{\isacharunderscore}{\kern0pt}Suc{\isacharparenright}{\kern0pt}%
\endisatagproof
{\isafoldproof}%
%
\isadelimproof
\isanewline
%
\endisadelimproof
\isanewline
\isacommand{lemma}\isamarkupfalse%
\ inc{\isacharunderscore}{\kern0pt}not{\isacharunderscore}{\kern0pt}eq{\isacharcolon}{\kern0pt}\ {\isachardoublequoteopen}valid\ b\ w\ {\isasymLongrightarrow}\ {\isacharparenleft}{\kern0pt}inc\ b\ w\ {\isacharequal}{\kern0pt}\ w{\isacharparenright}{\kern0pt}\ {\isacharequal}{\kern0pt}\ {\isacharparenleft}{\kern0pt}w\ {\isacharequal}{\kern0pt}\ {\isacharbrackleft}{\kern0pt}{\isacharbrackright}{\kern0pt}{\isacharparenright}{\kern0pt}{\isachardoublequoteclose}\isanewline
%
\isadelimproof
\ \ %
\endisadelimproof
%
\isatagproof
\isacommand{by}\isamarkupfalse%
\ {\isacharparenleft}{\kern0pt}induction\ w{\isacharparenright}{\kern0pt}\ auto%
\endisatagproof
{\isafoldproof}%
%
\isadelimproof
\isanewline
%
\endisadelimproof
%
\isadelimtheory
\isanewline
%
\endisadelimtheory
%
\isatagtheory
\isacommand{end}\isamarkupfalse%
%
\endisatagtheory
{\isafoldtheory}%
%
\isadelimtheory
%
\endisadelimtheory
%
\end{isabellebody}%
\endinput
%:%file=Encoding_Nat.tex%:%
%:%11=6%:%
%:%27=8%:%
%:%28=8%:%
%:%29=9%:%
%:%30=10%:%
%:%39=13%:%
%:%40=14%:%
%:%41=15%:%
%:%42=16%:%
%:%51=20%:%
%:%63=23%:%
%:%64=24%:%
%:%65=25%:%
%:%66=26%:%
%:%68=29%:%
%:%69=29%:%
%:%70=30%:%
%:%71=31%:%
%:%72=31%:%
%:%73=32%:%
%:%74=33%:%
%:%75=33%:%
%:%76=34%:%
%:%77=35%:%
%:%78=36%:%
%:%79=37%:%
%:%80=37%:%
%:%81=38%:%
%:%82=39%:%
%:%84=41%:%
%:%86=43%:%
%:%87=43%:%
%:%88=44%:%
%:%95=45%:%
%:%96=45%:%
%:%97=46%:%
%:%98=46%:%
%:%99=46%:%
%:%100=46%:%
%:%101=47%:%
%:%102=47%:%
%:%103=48%:%
%:%104=48%:%
%:%105=49%:%
%:%106=49%:%
%:%107=49%:%
%:%108=50%:%
%:%109=50%:%
%:%110=50%:%
%:%111=50%:%
%:%112=51%:%
%:%113=51%:%
%:%114=51%:%
%:%115=51%:%
%:%116=51%:%
%:%117=52%:%
%:%118=52%:%
%:%119=52%:%
%:%120=52%:%
%:%121=53%:%
%:%122=53%:%
%:%123=53%:%
%:%124=53%:%
%:%125=54%:%
%:%131=54%:%
%:%134=55%:%
%:%135=56%:%
%:%136=56%:%
%:%137=57%:%
%:%140=58%:%
%:%144=58%:%
%:%145=58%:%
%:%159=61%:%
%:%171=64%:%
%:%172=65%:%
%:%173=66%:%
%:%174=67%:%
%:%175=68%:%
%:%176=69%:%
%:%177=70%:%
%:%179=73%:%
%:%180=73%:%
%:%181=74%:%
%:%182=75%:%
%:%183=76%:%
%:%184=77%:%
%:%185=77%:%
%:%186=78%:%
%:%187=79%:%
%:%188=80%:%
%:%189=81%:%
%:%190=81%:%
%:%191=82%:%
%:%192=83%:%
%:%193=84%:%
%:%194=85%:%
%:%195=85%:%
%:%196=86%:%
%:%203=89%:%
%:%215=92%:%
%:%216=93%:%
%:%218=96%:%
%:%219=96%:%
%:%220=97%:%
%:%223=98%:%
%:%227=98%:%
%:%228=98%:%
%:%233=98%:%
%:%236=99%:%
%:%237=100%:%
%:%238=100%:%
%:%239=101%:%
%:%242=102%:%
%:%246=102%:%
%:%247=102%:%
%:%252=102%:%
%:%255=103%:%
%:%256=104%:%
%:%257=104%:%
%:%258=105%:%
%:%261=106%:%
%:%265=106%:%
%:%266=106%:%
%:%271=106%:%
%:%274=107%:%
%:%275=108%:%
%:%276=108%:%
%:%277=109%:%
%:%280=110%:%
%:%284=110%:%
%:%285=110%:%
%:%290=110%:%
%:%293=111%:%
%:%294=112%:%
%:%295=112%:%
%:%296=113%:%
%:%299=114%:%
%:%303=114%:%
%:%304=114%:%
%:%305=115%:%
%:%306=115%:%
%:%311=115%:%
%:%314=116%:%
%:%315=117%:%
%:%316=117%:%
%:%317=118%:%
%:%320=119%:%
%:%324=119%:%
%:%325=119%:%
%:%330=119%:%
%:%333=120%:%
%:%334=121%:%
%:%335=121%:%
%:%336=122%:%
%:%339=123%:%
%:%343=123%:%
%:%344=123%:%
%:%349=123%:%
%:%352=124%:%
%:%353=125%:%
%:%354=125%:%
%:%355=126%:%
%:%358=127%:%
%:%362=127%:%
%:%363=127%:%
%:%368=127%:%
%:%371=128%:%
%:%372=129%:%
%:%373=129%:%
%:%374=130%:%
%:%377=131%:%
%:%381=131%:%
%:%382=131%:%
%:%387=131%:%
%:%390=132%:%
%:%391=133%:%
%:%392=133%:%
%:%393=134%:%
%:%400=135%:%
%:%401=135%:%
%:%402=136%:%
%:%403=136%:%
%:%404=137%:%
%:%405=137%:%
%:%406=138%:%
%:%407=138%:%
%:%408=138%:%
%:%409=139%:%
%:%415=139%:%
%:%418=140%:%
%:%419=141%:%
%:%420=141%:%
%:%421=142%:%
%:%424=143%:%
%:%428=143%:%
%:%429=143%:%
%:%434=143%:%
%:%437=144%:%
%:%438=145%:%
%:%439=145%:%
%:%440=146%:%
%:%443=147%:%
%:%447=147%:%
%:%448=147%:%
%:%453=147%:%
%:%456=148%:%
%:%457=149%:%
%:%458=149%:%
%:%459=150%:%
%:%466=151%:%
%:%467=151%:%
%:%468=152%:%
%:%469=152%:%
%:%470=153%:%
%:%471=153%:%
%:%472=154%:%
%:%473=154%:%
%:%474=155%:%
%:%475=156%:%
%:%476=156%:%
%:%477=156%:%
%:%478=157%:%
%:%479=157%:%
%:%480=157%:%
%:%481=157%:%
%:%482=157%:%
%:%483=158%:%
%:%489=158%:%
%:%492=159%:%
%:%493=160%:%
%:%494=160%:%
%:%495=161%:%
%:%498=162%:%
%:%502=162%:%
%:%503=162%:%
%:%508=162%:%
%:%511=163%:%
%:%512=164%:%
%:%513=164%:%
%:%514=165%:%
%:%521=166%:%
%:%522=166%:%
%:%523=167%:%
%:%524=167%:%
%:%525=168%:%
%:%526=168%:%
%:%527=169%:%
%:%528=169%:%
%:%529=170%:%
%:%530=170%:%
%:%531=171%:%
%:%532=171%:%
%:%533=172%:%
%:%534=172%:%
%:%535=173%:%
%:%536=173%:%
%:%537=174%:%
%:%538=174%:%
%:%539=175%:%
%:%540=175%:%
%:%541=175%:%
%:%542=176%:%
%:%543=176%:%
%:%544=176%:%
%:%545=177%:%
%:%546=177%:%
%:%547=177%:%
%:%548=178%:%
%:%549=178%:%
%:%550=179%:%
%:%556=179%:%
%:%559=180%:%
%:%560=181%:%
%:%561=181%:%
%:%562=182%:%
%:%569=183%:%
%:%570=183%:%
%:%571=184%:%
%:%572=184%:%
%:%573=185%:%
%:%574=185%:%
%:%575=186%:%
%:%576=186%:%
%:%577=187%:%
%:%578=187%:%
%:%579=188%:%
%:%580=188%:%
%:%581=189%:%
%:%582=189%:%
%:%583=190%:%
%:%584=190%:%
%:%585=191%:%
%:%586=191%:%
%:%587=192%:%
%:%588=192%:%
%:%589=193%:%
%:%590=193%:%
%:%591=193%:%
%:%592=194%:%
%:%593=194%:%
%:%594=194%:%
%:%595=194%:%
%:%596=195%:%
%:%597=195%:%
%:%598=195%:%
%:%599=196%:%
%:%600=196%:%
%:%601=196%:%
%:%602=196%:%
%:%603=196%:%
%:%604=197%:%
%:%605=197%:%
%:%606=197%:%
%:%607=198%:%
%:%608=198%:%
%:%609=199%:%
%:%615=199%:%
%:%618=200%:%
%:%619=201%:%
%:%620=201%:%
%:%621=202%:%
%:%624=203%:%
%:%628=203%:%
%:%629=203%:%
%:%638=206%:%
%:%640=209%:%
%:%641=209%:%
%:%642=210%:%
%:%645=211%:%
%:%649=211%:%
%:%650=211%:%
%:%664=214%:%
%:%676=217%:%
%:%677=218%:%
%:%678=219%:%
%:%679=220%:%
%:%680=221%:%
%:%682=224%:%
%:%683=224%:%
%:%684=225%:%
%:%685=226%:%
%:%686=227%:%
%:%687=228%:%
%:%688=228%:%
%:%689=229%:%
%:%692=230%:%
%:%696=230%:%
%:%697=230%:%
%:%702=230%:%
%:%705=231%:%
%:%706=232%:%
%:%707=232%:%
%:%708=233%:%
%:%711=234%:%
%:%715=234%:%
%:%716=234%:%
%:%725=237%:%
%:%726=238%:%
%:%727=239%:%
%:%729=242%:%
%:%730=242%:%
%:%731=243%:%
%:%738=244%:%
%:%739=244%:%
%:%740=245%:%
%:%741=245%:%
%:%742=245%:%
%:%743=245%:%
%:%744=246%:%
%:%745=246%:%
%:%746=247%:%
%:%747=247%:%
%:%748=248%:%
%:%749=248%:%
%:%750=248%:%
%:%751=249%:%
%:%752=249%:%
%:%753=250%:%
%:%754=250%:%
%:%755=251%:%
%:%756=251%:%
%:%757=252%:%
%:%758=252%:%
%:%759=252%:%
%:%760=253%:%
%:%761=253%:%
%:%762=253%:%
%:%763=253%:%
%:%764=253%:%
%:%765=254%:%
%:%766=254%:%
%:%767=254%:%
%:%768=254%:%
%:%769=254%:%
%:%770=255%:%
%:%771=255%:%
%:%772=255%:%
%:%773=255%:%
%:%774=256%:%
%:%775=256%:%
%:%776=256%:%
%:%777=256%:%
%:%778=257%:%
%:%779=257%:%
%:%780=257%:%
%:%781=257%:%
%:%782=258%:%
%:%783=258%:%
%:%784=259%:%
%:%785=259%:%
%:%786=260%:%
%:%787=260%:%
%:%788=261%:%
%:%789=261%:%
%:%790=261%:%
%:%791=261%:%
%:%792=262%:%
%:%793=262%:%
%:%794=262%:%
%:%795=262%:%
%:%796=263%:%
%:%797=263%:%
%:%798=263%:%
%:%799=263%:%
%:%800=264%:%
%:%801=264%:%
%:%802=264%:%
%:%803=264%:%
%:%804=265%:%
%:%805=265%:%
%:%806=265%:%
%:%807=266%:%
%:%808=266%:%
%:%809=266%:%
%:%810=266%:%
%:%811=267%:%
%:%812=267%:%
%:%813=268%:%
%:%819=268%:%
%:%822=269%:%
%:%823=270%:%
%:%824=270%:%
%:%825=271%:%
%:%828=272%:%
%:%832=272%:%
%:%833=272%:%
%:%834=273%:%
%:%835=273%:%
%:%840=273%:%
%:%843=274%:%
%:%844=275%:%
%:%845=275%:%
%:%848=276%:%
%:%852=276%:%
%:%853=276%:%
%:%858=276%:%
%:%863=277%:%
%:%868=278%:%

%
\begin{isabellebody}%
\setisabellecontext{Code{\isacharunderscore}{\kern0pt}Word{\isacharunderscore}{\kern0pt}Dist}%
%
\isadelimdocument
%
\endisadelimdocument
%
\isatagdocument
%
\isamarkupsection{A Generalized Distance Measure%
}
\isamarkuptrue%
%
\endisatagdocument
{\isafolddocument}%
%
\isadelimdocument
%
\endisadelimdocument
%
\isadelimtheory
%
\endisadelimtheory
%
\isatagtheory
\isacommand{theory}\isamarkupfalse%
\ Code{\isacharunderscore}{\kern0pt}Word{\isacharunderscore}{\kern0pt}Dist\isanewline
\ \ \isakeyword{imports}\ Encoding{\isacharunderscore}{\kern0pt}Nat\isanewline
\isakeyword{begin}%
\endisatagtheory
{\isafoldtheory}%
%
\isadelimtheory
%
\endisadelimtheory
%
\begin{isamarkuptext}%
In the case of the reflected binary code (RBC) it is sufficient
  to use the Hamming distance to express the property, because there are
  only two distinct digits so that a bitflip naturally always corresponds
  to a distance of 1.%
\end{isamarkuptext}\isamarkuptrue%
%
\isadelimdocument
%
\endisadelimdocument
%
\isatagdocument
%
\isamarkupsubsection{Distance of Digits%
}
\isamarkuptrue%
%
\endisatagdocument
{\isafolddocument}%
%
\isadelimdocument
%
\endisadelimdocument
%
\begin{isamarkuptext}%
We can interpret a bitflip as an increment modulo 2, which is why for the
  distance of digits it appears as a natural generalization to choose the
  amount of required increments.
Mathematically, the distance $d(x,y)$ should be $y-x$ (mod $b$).
For example we have $d(0,1) = d(1,0) = 1$ in the binary numeral system.%
\end{isamarkuptext}\isamarkuptrue%
\isacommand{definition}\isamarkupfalse%
\ dist{\isadigit{1}}\ {\isacharcolon}{\kern0pt}{\isacharcolon}{\kern0pt}\ {\isachardoublequoteopen}base\ {\isasymRightarrow}\ nat\ {\isasymRightarrow}\ nat\ {\isasymRightarrow}\ nat{\isachardoublequoteclose}\ \isakeyword{where}\isanewline
\ \ {\isachardoublequoteopen}dist{\isadigit{1}}\ b\ x\ y\ {\isasymequiv}\ if\ x{\isasymle}y\ then\ y{\isacharminus}{\kern0pt}x\ else\ b{\isacharplus}{\kern0pt}y{\isacharminus}{\kern0pt}x{\isachardoublequoteclose}%
\begin{isamarkuptext}%
Note that the distance of digits is in general asymmetric, so that it is
  in paticular not a metric. However, this is not an issue and in fact the
  most appropriate generalization, partly due to the next lemma:%
\end{isamarkuptext}\isamarkuptrue%
\isacommand{lemma}\isamarkupfalse%
\ dist{\isadigit{1}}{\isacharunderscore}{\kern0pt}eq{\isacharcolon}{\kern0pt}\isanewline
\ \ {\isachardoublequoteopen}{\isasymlbrakk}x\ {\isacharless}{\kern0pt}\ b{\isacharsemicolon}{\kern0pt}\ y\ {\isacharless}{\kern0pt}\ b{\isacharsemicolon}{\kern0pt}\ dist{\isadigit{1}}\ b\ x\ y\ {\isacharequal}{\kern0pt}\ {\isadigit{0}}{\isasymrbrakk}\ {\isasymLongrightarrow}\ x\ {\isacharequal}{\kern0pt}\ y{\isachardoublequoteclose}\isanewline
%
\isadelimproof
\ \ %
\endisadelimproof
%
\isatagproof
\isacommand{by}\isamarkupfalse%
\ {\isacharparenleft}{\kern0pt}auto\ simp\ add{\isacharcolon}{\kern0pt}\ dist{\isadigit{1}}{\isacharunderscore}{\kern0pt}def\ split{\isacharcolon}{\kern0pt}\ if{\isacharunderscore}{\kern0pt}splits{\isacharparenright}{\kern0pt}%
\endisatagproof
{\isafoldproof}%
%
\isadelimproof
\isanewline
%
\endisadelimproof
\isanewline
\isacommand{lemma}\isamarkupfalse%
\ dist{\isadigit{1}}{\isacharunderscore}{\kern0pt}{\isadigit{0}}{\isacharcolon}{\kern0pt}\isanewline
\ \ {\isachardoublequoteopen}dist{\isadigit{1}}\ b\ x\ x\ {\isacharequal}{\kern0pt}\ {\isadigit{0}}{\isachardoublequoteclose}\isanewline
%
\isadelimproof
\ \ %
\endisadelimproof
%
\isatagproof
\isacommand{by}\isamarkupfalse%
\ {\isacharparenleft}{\kern0pt}auto\ simp\ add{\isacharcolon}{\kern0pt}\ dist{\isadigit{1}}{\isacharunderscore}{\kern0pt}def{\isacharparenright}{\kern0pt}%
\endisatagproof
{\isafoldproof}%
%
\isadelimproof
\isanewline
%
\endisadelimproof
\isanewline
\isacommand{lemma}\isamarkupfalse%
\ dist{\isadigit{1}}{\isacharunderscore}{\kern0pt}ge{\isadigit{1}}{\isacharcolon}{\kern0pt}\isanewline
\ \ {\isachardoublequoteopen}{\isasymlbrakk}x\ {\isacharless}{\kern0pt}\ b{\isacharsemicolon}{\kern0pt}\ y\ {\isacharless}{\kern0pt}\ b{\isacharsemicolon}{\kern0pt}\ x{\isasymnoteq}y{\isasymrbrakk}\ {\isasymLongrightarrow}\ dist{\isadigit{1}}\ b\ x\ y\ {\isasymge}\ {\isadigit{1}}{\isachardoublequoteclose}\isanewline
%
\isadelimproof
\ \ %
\endisadelimproof
%
\isatagproof
\isacommand{using}\isamarkupfalse%
\ dist{\isadigit{1}}{\isacharunderscore}{\kern0pt}eq\ \isacommand{by}\isamarkupfalse%
\ fastforce%
\endisatagproof
{\isafoldproof}%
%
\isadelimproof
\isanewline
%
\endisadelimproof
\isanewline
\isacommand{lemma}\isamarkupfalse%
\ dist{\isadigit{1}}{\isacharunderscore}{\kern0pt}elim{\isacharunderscore}{\kern0pt}{\isadigit{1}}{\isacharcolon}{\kern0pt}\isanewline
\ \ {\isachardoublequoteopen}{\isasymlbrakk}x\ {\isacharless}{\kern0pt}\ b{\isacharsemicolon}{\kern0pt}\ y\ {\isacharless}{\kern0pt}\ b{\isasymrbrakk}\ {\isasymLongrightarrow}\ {\isacharparenleft}{\kern0pt}dist{\isadigit{1}}\ b\ x\ y\ {\isacharplus}{\kern0pt}\ x{\isacharparenright}{\kern0pt}\ mod\ b\ {\isacharequal}{\kern0pt}\ y{\isachardoublequoteclose}\isanewline
%
\isadelimproof
\ \ %
\endisadelimproof
%
\isatagproof
\isacommand{by}\isamarkupfalse%
\ {\isacharparenleft}{\kern0pt}auto\ simp\ add{\isacharcolon}{\kern0pt}\ dist{\isadigit{1}}{\isacharunderscore}{\kern0pt}def{\isacharparenright}{\kern0pt}%
\endisatagproof
{\isafoldproof}%
%
\isadelimproof
\isanewline
%
\endisadelimproof
\isanewline
\isacommand{lemma}\isamarkupfalse%
\ dist{\isadigit{1}}{\isacharunderscore}{\kern0pt}elim{\isacharunderscore}{\kern0pt}{\isadigit{2}}{\isacharcolon}{\kern0pt}\isanewline
\ \ {\isachardoublequoteopen}{\isasymlbrakk}x\ {\isacharless}{\kern0pt}\ b{\isacharsemicolon}{\kern0pt}\ y\ {\isacharless}{\kern0pt}\ b{\isasymrbrakk}\ {\isasymLongrightarrow}\ dist{\isadigit{1}}\ b\ x\ {\isacharparenleft}{\kern0pt}x{\isacharplus}{\kern0pt}y{\isacharparenright}{\kern0pt}\ {\isacharequal}{\kern0pt}\ y{\isachardoublequoteclose}\isanewline
%
\isadelimproof
\ \ %
\endisadelimproof
%
\isatagproof
\isacommand{by}\isamarkupfalse%
\ {\isacharparenleft}{\kern0pt}auto\ simp\ add{\isacharcolon}{\kern0pt}\ dist{\isadigit{1}}{\isacharunderscore}{\kern0pt}def{\isacharparenright}{\kern0pt}%
\endisatagproof
{\isafoldproof}%
%
\isadelimproof
\isanewline
%
\endisadelimproof
\isanewline
\isacommand{lemma}\isamarkupfalse%
\ dist{\isadigit{1}}{\isacharunderscore}{\kern0pt}mod{\isacharunderscore}{\kern0pt}Suc{\isacharcolon}{\kern0pt}\isanewline
\ \ {\isachardoublequoteopen}{\isasymlbrakk}x\ {\isacharless}{\kern0pt}\ b{\isacharsemicolon}{\kern0pt}\ y\ {\isacharless}{\kern0pt}\ b{\isasymrbrakk}\ {\isasymLongrightarrow}\ dist{\isadigit{1}}\ b\ x\ {\isacharparenleft}{\kern0pt}Suc\ y\ mod\ b{\isacharparenright}{\kern0pt}\ {\isacharequal}{\kern0pt}\ Suc\ {\isacharparenleft}{\kern0pt}dist{\isadigit{1}}\ b\ x\ y{\isacharparenright}{\kern0pt}\ mod\ b{\isachardoublequoteclose}\isanewline
%
\isadelimproof
\ \ %
\endisadelimproof
%
\isatagproof
\isacommand{by}\isamarkupfalse%
\ {\isacharparenleft}{\kern0pt}auto\ simp\ add{\isacharcolon}{\kern0pt}\ dist{\isadigit{1}}{\isacharunderscore}{\kern0pt}def\ mod{\isacharunderscore}{\kern0pt}Suc{\isacharparenright}{\kern0pt}%
\endisatagproof
{\isafoldproof}%
%
\isadelimproof
\isanewline
%
\endisadelimproof
\isanewline
\isacommand{lemma}\isamarkupfalse%
\ dist{\isadigit{1}}{\isacharunderscore}{\kern0pt}Suc{\isacharcolon}{\kern0pt}\isanewline
\ \ {\isachardoublequoteopen}{\isasymlbrakk}{\isadigit{2}}\ {\isasymle}\ b{\isacharsemicolon}{\kern0pt}\ x\ {\isacharless}{\kern0pt}\ b{\isasymrbrakk}\ {\isasymLongrightarrow}\ dist{\isadigit{1}}\ b\ x\ {\isacharparenleft}{\kern0pt}Suc\ x\ mod\ b{\isacharparenright}{\kern0pt}\ {\isacharequal}{\kern0pt}\ {\isadigit{1}}{\isachardoublequoteclose}\isanewline
%
\isadelimproof
\ \ %
\endisadelimproof
%
\isatagproof
\isacommand{by}\isamarkupfalse%
\ {\isacharparenleft}{\kern0pt}simp\ add{\isacharcolon}{\kern0pt}\ dist{\isadigit{1}}{\isacharunderscore}{\kern0pt}{\isadigit{0}}\ dist{\isadigit{1}}{\isacharunderscore}{\kern0pt}mod{\isacharunderscore}{\kern0pt}Suc{\isacharparenright}{\kern0pt}%
\endisatagproof
{\isafoldproof}%
%
\isadelimproof
\isanewline
%
\endisadelimproof
\isanewline
\isacommand{lemma}\isamarkupfalse%
\ dist{\isadigit{1}}{\isacharunderscore}{\kern0pt}asym{\isacharcolon}{\kern0pt}\isanewline
\ \ {\isachardoublequoteopen}{\isasymlbrakk}x\ {\isacharless}{\kern0pt}\ b{\isacharsemicolon}{\kern0pt}\ y\ {\isacharless}{\kern0pt}\ b{\isasymrbrakk}\ {\isasymLongrightarrow}\ {\isacharparenleft}{\kern0pt}dist{\isadigit{1}}\ b\ x\ y\ {\isacharplus}{\kern0pt}\ dist{\isadigit{1}}\ b\ y\ x{\isacharparenright}{\kern0pt}\ mod\ b\ {\isacharequal}{\kern0pt}\ {\isadigit{0}}{\isachardoublequoteclose}\isanewline
%
\isadelimproof
\ \ %
\endisadelimproof
%
\isatagproof
\isacommand{by}\isamarkupfalse%
\ {\isacharparenleft}{\kern0pt}auto\ simp\ add{\isacharcolon}{\kern0pt}\ dist{\isadigit{1}}{\isacharunderscore}{\kern0pt}def{\isacharparenright}{\kern0pt}%
\endisatagproof
{\isafoldproof}%
%
\isadelimproof
\isanewline
%
\endisadelimproof
\isanewline
\isacommand{lemma}\isamarkupfalse%
\ dist{\isadigit{1}}{\isacharunderscore}{\kern0pt}valid{\isacharcolon}{\kern0pt}\isanewline
\ \ {\isachardoublequoteopen}{\isasymlbrakk}x\ {\isacharless}{\kern0pt}\ b{\isacharsemicolon}{\kern0pt}\ y\ {\isacharless}{\kern0pt}\ b{\isasymrbrakk}\ {\isasymLongrightarrow}\ dist{\isadigit{1}}\ b\ x\ y\ {\isacharless}{\kern0pt}\ b{\isachardoublequoteclose}\isanewline
%
\isadelimproof
\ \ %
\endisadelimproof
%
\isatagproof
\isacommand{by}\isamarkupfalse%
\ {\isacharparenleft}{\kern0pt}auto\ simp\ add{\isacharcolon}{\kern0pt}\ dist{\isadigit{1}}{\isacharunderscore}{\kern0pt}def{\isacharparenright}{\kern0pt}%
\endisatagproof
{\isafoldproof}%
%
\isadelimproof
\isanewline
%
\endisadelimproof
\isanewline
\isanewline
\isacommand{lemma}\isamarkupfalse%
\ dist{\isadigit{1}}{\isacharunderscore}{\kern0pt}distr{\isacharcolon}{\kern0pt}\isanewline
\ \ {\isachardoublequoteopen}{\isasymlbrakk}x\ {\isacharless}{\kern0pt}\ b{\isacharsemicolon}{\kern0pt}\ y\ {\isacharless}{\kern0pt}\ b{\isacharsemicolon}{\kern0pt}\ z\ {\isacharless}{\kern0pt}\ b{\isasymrbrakk}\ {\isasymLongrightarrow}\ dist{\isadigit{1}}\ b\ {\isacharparenleft}{\kern0pt}dist{\isadigit{1}}\ b\ x\ y{\isacharparenright}{\kern0pt}\ {\isacharparenleft}{\kern0pt}dist{\isadigit{1}}\ b\ x\ z{\isacharparenright}{\kern0pt}\ {\isacharequal}{\kern0pt}\ dist{\isadigit{1}}\ b\ y\ z{\isachardoublequoteclose}\isanewline
%
\isadelimproof
\ \ %
\endisadelimproof
%
\isatagproof
\isacommand{by}\isamarkupfalse%
\ {\isacharparenleft}{\kern0pt}auto\ simp\ add{\isacharcolon}{\kern0pt}\ dist{\isadigit{1}}{\isacharunderscore}{\kern0pt}def{\isacharparenright}{\kern0pt}%
\endisatagproof
{\isafoldproof}%
%
\isadelimproof
\isanewline
%
\endisadelimproof
\isanewline
\isacommand{lemma}\isamarkupfalse%
\ dist{\isadigit{1}}{\isacharunderscore}{\kern0pt}distr{\isadigit{2}}{\isacharcolon}{\kern0pt}\isanewline
\ \ {\isachardoublequoteopen}{\isasymlbrakk}x\ {\isacharless}{\kern0pt}\ b{\isacharsemicolon}{\kern0pt}\ y\ {\isacharless}{\kern0pt}\ b{\isacharsemicolon}{\kern0pt}\ z\ {\isacharless}{\kern0pt}\ b{\isasymrbrakk}\ {\isasymLongrightarrow}\ dist{\isadigit{1}}\ b\ {\isacharparenleft}{\kern0pt}dist{\isadigit{1}}\ b\ x\ z{\isacharparenright}{\kern0pt}\ {\isacharparenleft}{\kern0pt}dist{\isadigit{1}}\ b\ y\ z{\isacharparenright}{\kern0pt}\ {\isacharequal}{\kern0pt}\ dist{\isadigit{1}}\ b\ y\ x{\isachardoublequoteclose}\isanewline
%
\isadelimproof
\ \ %
\endisadelimproof
%
\isatagproof
\isacommand{by}\isamarkupfalse%
\ {\isacharparenleft}{\kern0pt}auto\ simp\ add{\isacharcolon}{\kern0pt}\ dist{\isadigit{1}}{\isacharunderscore}{\kern0pt}def{\isacharparenright}{\kern0pt}%
\endisatagproof
{\isafoldproof}%
%
\isadelimproof
%
\endisadelimproof
%
\isadelimdocument
%
\endisadelimdocument
%
\isatagdocument
%
\isamarkupsubsection{(Hamming-) Distance between Words%
}
\isamarkuptrue%
%
\endisatagdocument
{\isafolddocument}%
%
\isadelimdocument
%
\endisadelimdocument
%
\begin{isamarkuptext}%
The total distance between two words of equal length is then defined as
  the sum of component-wise distances.
Note that the Hamming distance is equivalent to this definition for $b=2$
  and is in general a lower bound.%
\end{isamarkuptext}\isamarkuptrue%
\isacommand{fun}\isamarkupfalse%
\ hamming\ {\isacharcolon}{\kern0pt}{\isacharcolon}{\kern0pt}\ {\isachardoublequoteopen}word\ {\isasymRightarrow}\ word\ {\isasymRightarrow}\ nat{\isachardoublequoteclose}\ \isakeyword{where}\isanewline
\ \ {\isachardoublequoteopen}hamming\ {\isacharbrackleft}{\kern0pt}{\isacharbrackright}{\kern0pt}\ {\isacharbrackleft}{\kern0pt}{\isacharbrackright}{\kern0pt}\ {\isacharequal}{\kern0pt}\ {\isadigit{0}}{\isachardoublequoteclose}\isanewline
{\isacharbar}{\kern0pt}\ {\isachardoublequoteopen}hamming\ {\isacharparenleft}{\kern0pt}a{\isacharhash}{\kern0pt}v{\isacharparenright}{\kern0pt}\ {\isacharparenleft}{\kern0pt}b{\isacharhash}{\kern0pt}w{\isacharparenright}{\kern0pt}\ {\isacharequal}{\kern0pt}\ {\isacharparenleft}{\kern0pt}if\ a{\isasymnoteq}b\ then\ {\isadigit{1}}\ else\ {\isadigit{0}}{\isacharparenright}{\kern0pt}\ {\isacharplus}{\kern0pt}\ hamming\ v\ w{\isachardoublequoteclose}%
\begin{isamarkuptext}%
The Hamming distance is only defined in the case of equal word length.
In the following definition we assume leading zeroes if the word length
  is not equal:%
\end{isamarkuptext}\isamarkuptrue%
\isacommand{fun}\isamarkupfalse%
\ dist\ {\isacharcolon}{\kern0pt}{\isacharcolon}{\kern0pt}\ {\isachardoublequoteopen}base\ {\isasymRightarrow}\ word\ {\isasymRightarrow}\ word\ {\isasymRightarrow}\ nat{\isachardoublequoteclose}\ \isakeyword{where}\isanewline
\ \ {\isachardoublequoteopen}dist\ {\isacharunderscore}{\kern0pt}\ {\isacharbrackleft}{\kern0pt}{\isacharbrackright}{\kern0pt}\ {\isacharbrackleft}{\kern0pt}{\isacharbrackright}{\kern0pt}\ {\isacharequal}{\kern0pt}\ {\isadigit{0}}{\isachardoublequoteclose}\isanewline
{\isacharbar}{\kern0pt}\ {\isachardoublequoteopen}dist\ b\ {\isacharparenleft}{\kern0pt}x{\isacharhash}{\kern0pt}xs{\isacharparenright}{\kern0pt}\ {\isacharbrackleft}{\kern0pt}{\isacharbrackright}{\kern0pt}\ {\isacharequal}{\kern0pt}\ dist{\isadigit{1}}\ b\ x\ {\isadigit{0}}\ {\isacharplus}{\kern0pt}\ dist\ b\ xs\ {\isacharbrackleft}{\kern0pt}{\isacharbrackright}{\kern0pt}{\isachardoublequoteclose}\isanewline
{\isacharbar}{\kern0pt}\ {\isachardoublequoteopen}dist\ b\ {\isacharbrackleft}{\kern0pt}{\isacharbrackright}{\kern0pt}\ {\isacharparenleft}{\kern0pt}y{\isacharhash}{\kern0pt}ys{\isacharparenright}{\kern0pt}\ {\isacharequal}{\kern0pt}\ dist{\isadigit{1}}\ b\ {\isadigit{0}}\ y\ {\isacharplus}{\kern0pt}\ dist\ b\ {\isacharbrackleft}{\kern0pt}{\isacharbrackright}{\kern0pt}\ ys{\isachardoublequoteclose}\isanewline
{\isacharbar}{\kern0pt}\ {\isachardoublequoteopen}dist\ b\ {\isacharparenleft}{\kern0pt}x{\isacharhash}{\kern0pt}xs{\isacharparenright}{\kern0pt}\ {\isacharparenleft}{\kern0pt}y{\isacharhash}{\kern0pt}ys{\isacharparenright}{\kern0pt}\ {\isacharequal}{\kern0pt}\ dist{\isadigit{1}}\ b\ x\ y\ {\isacharplus}{\kern0pt}\ dist\ b\ xs\ ys{\isachardoublequoteclose}\isanewline
\isanewline
\isacommand{lemma}\isamarkupfalse%
\ dist{\isacharunderscore}{\kern0pt}{\isadigit{0}}{\isacharcolon}{\kern0pt}\isanewline
\ \ {\isachardoublequoteopen}dist\ b\ w\ w\ {\isacharequal}{\kern0pt}\ {\isadigit{0}}{\isachardoublequoteclose}\isanewline
%
\isadelimproof
\ \ %
\endisadelimproof
%
\isatagproof
\isacommand{apply}\isamarkupfalse%
\ {\isacharparenleft}{\kern0pt}induction\ w{\isacharparenright}{\kern0pt}\isanewline
\ \ \isacommand{by}\isamarkupfalse%
\ {\isacharparenleft}{\kern0pt}auto\ simp\ add{\isacharcolon}{\kern0pt}\ dist{\isadigit{1}}{\isacharunderscore}{\kern0pt}{\isadigit{0}}{\isacharparenright}{\kern0pt}%
\endisatagproof
{\isafoldproof}%
%
\isadelimproof
\isanewline
%
\endisadelimproof
\isanewline
\isacommand{lemma}\isamarkupfalse%
\ dist{\isacharunderscore}{\kern0pt}eq{\isacharcolon}{\kern0pt}\isanewline
\ \ {\isachardoublequoteopen}{\isasymlbrakk}valid\ b\ v{\isacharsemicolon}{\kern0pt}\ valid\ b\ w{\isacharsemicolon}{\kern0pt}\ length\ v{\isacharequal}{\kern0pt}length\ w{\isacharsemicolon}{\kern0pt}\ dist\ b\ v\ w\ {\isacharequal}{\kern0pt}\ {\isadigit{0}}{\isasymrbrakk}\ {\isasymLongrightarrow}\ v\ {\isacharequal}{\kern0pt}\ w{\isachardoublequoteclose}\isanewline
%
\isadelimproof
\ \ %
\endisadelimproof
%
\isatagproof
\isacommand{apply}\isamarkupfalse%
\ {\isacharparenleft}{\kern0pt}induction\ b\ v\ w\ rule{\isacharcolon}{\kern0pt}\ dist{\isachardot}{\kern0pt}induct{\isacharparenright}{\kern0pt}\isanewline
\ \ \isacommand{by}\isamarkupfalse%
\ {\isacharparenleft}{\kern0pt}auto\ simp\ add{\isacharcolon}{\kern0pt}\ dist{\isadigit{1}}{\isacharunderscore}{\kern0pt}eq{\isacharparenright}{\kern0pt}%
\endisatagproof
{\isafoldproof}%
%
\isadelimproof
\isanewline
%
\endisadelimproof
\isanewline
\isacommand{lemma}\isamarkupfalse%
\ dist{\isacharunderscore}{\kern0pt}posd{\isacharcolon}{\kern0pt}\isanewline
\ \ {\isachardoublequoteopen}{\isasymlbrakk}valid\ b\ v{\isacharsemicolon}{\kern0pt}\ valid\ b\ w{\isacharsemicolon}{\kern0pt}\ length\ v{\isacharequal}{\kern0pt}length\ w{\isasymrbrakk}\ {\isasymLongrightarrow}\ {\isacharparenleft}{\kern0pt}dist\ b\ v\ w\ {\isacharequal}{\kern0pt}\ {\isadigit{0}}{\isacharparenright}{\kern0pt}\ {\isacharequal}{\kern0pt}\ {\isacharparenleft}{\kern0pt}v\ {\isacharequal}{\kern0pt}\ w{\isacharparenright}{\kern0pt}{\isachardoublequoteclose}\isanewline
%
\isadelimproof
\ \ %
\endisadelimproof
%
\isatagproof
\isacommand{using}\isamarkupfalse%
\ dist{\isacharunderscore}{\kern0pt}{\isadigit{0}}\ dist{\isacharunderscore}{\kern0pt}eq\ \isacommand{by}\isamarkupfalse%
\ auto%
\endisatagproof
{\isafoldproof}%
%
\isadelimproof
\isanewline
%
\endisadelimproof
\isanewline
\isacommand{lemma}\isamarkupfalse%
\ hamming{\isacharunderscore}{\kern0pt}posd{\isacharcolon}{\kern0pt}\isanewline
\ \ {\isachardoublequoteopen}length\ v{\isacharequal}{\kern0pt}length\ w\ {\isasymLongrightarrow}\ {\isacharparenleft}{\kern0pt}hamming\ v\ w\ {\isacharequal}{\kern0pt}\ {\isadigit{0}}{\isacharparenright}{\kern0pt}\ {\isacharequal}{\kern0pt}\ {\isacharparenleft}{\kern0pt}v\ {\isacharequal}{\kern0pt}\ w{\isacharparenright}{\kern0pt}{\isachardoublequoteclose}\isanewline
%
\isadelimproof
\ \ %
\endisadelimproof
%
\isatagproof
\isacommand{by}\isamarkupfalse%
\ {\isacharparenleft}{\kern0pt}induction\ v\ w\ rule{\isacharcolon}{\kern0pt}\ hamming{\isachardot}{\kern0pt}induct{\isacharparenright}{\kern0pt}\ auto%
\endisatagproof
{\isafoldproof}%
%
\isadelimproof
\isanewline
%
\endisadelimproof
\isanewline
\isacommand{lemma}\isamarkupfalse%
\ hamming{\isacharunderscore}{\kern0pt}symm{\isacharcolon}{\kern0pt}\isanewline
\ \ {\isachardoublequoteopen}length\ v{\isacharequal}{\kern0pt}length\ w\ {\isasymLongrightarrow}\ hamming\ v\ w\ {\isacharequal}{\kern0pt}\ hamming\ w\ v{\isachardoublequoteclose}\isanewline
%
\isadelimproof
\ \ %
\endisadelimproof
%
\isatagproof
\isacommand{by}\isamarkupfalse%
\ {\isacharparenleft}{\kern0pt}induction\ v\ w\ rule{\isacharcolon}{\kern0pt}\ hamming{\isachardot}{\kern0pt}induct{\isacharparenright}{\kern0pt}\ auto%
\endisatagproof
{\isafoldproof}%
%
\isadelimproof
\isanewline
%
\endisadelimproof
\isanewline
\isacommand{theorem}\isamarkupfalse%
\ hamming{\isacharunderscore}{\kern0pt}dist{\isacharcolon}{\kern0pt}\isanewline
\ \ {\isachardoublequoteopen}{\isasymlbrakk}valid\ b\ v{\isacharsemicolon}{\kern0pt}\ valid\ b\ w{\isacharsemicolon}{\kern0pt}\ length\ v{\isacharequal}{\kern0pt}length\ w{\isasymrbrakk}\ {\isasymLongrightarrow}\ hamming\ v\ w\ {\isasymle}\ dist\ b\ v\ w{\isachardoublequoteclose}\isanewline
%
\isadelimproof
\ \ %
\endisadelimproof
%
\isatagproof
\isacommand{apply}\isamarkupfalse%
\ {\isacharparenleft}{\kern0pt}induction\ b\ v\ w\ rule{\isacharcolon}{\kern0pt}\ dist{\isachardot}{\kern0pt}induct{\isacharparenright}{\kern0pt}\isanewline
\ \ \ \ \ \isacommand{apply}\isamarkupfalse%
\ auto\isanewline
\ \ \isacommand{using}\isamarkupfalse%
\ dist{\isadigit{1}}{\isacharunderscore}{\kern0pt}ge{\isadigit{1}}\ \isacommand{by}\isamarkupfalse%
\ fastforce%
\endisatagproof
{\isafoldproof}%
%
\isadelimproof
\isanewline
%
\endisadelimproof
%
\isadelimtheory
\isanewline
%
\endisadelimtheory
%
\isatagtheory
\isacommand{end}\isamarkupfalse%
%
\endisatagtheory
{\isafoldtheory}%
%
\isadelimtheory
%
\endisadelimtheory
%
\end{isabellebody}%
\endinput
%:%file=Code_Word_Dist.tex%:%
%:%11=6%:%
%:%27=8%:%
%:%28=8%:%
%:%29=9%:%
%:%30=10%:%
%:%39=13%:%
%:%40=14%:%
%:%41=15%:%
%:%42=16%:%
%:%51=20%:%
%:%63=23%:%
%:%64=24%:%
%:%65=25%:%
%:%66=26%:%
%:%67=27%:%
%:%69=30%:%
%:%70=30%:%
%:%71=31%:%
%:%73=34%:%
%:%74=35%:%
%:%75=36%:%
%:%77=39%:%
%:%78=39%:%
%:%79=40%:%
%:%82=41%:%
%:%86=41%:%
%:%87=41%:%
%:%92=41%:%
%:%95=42%:%
%:%96=43%:%
%:%97=43%:%
%:%98=44%:%
%:%101=45%:%
%:%105=45%:%
%:%106=45%:%
%:%111=45%:%
%:%114=46%:%
%:%115=47%:%
%:%116=47%:%
%:%117=48%:%
%:%120=49%:%
%:%124=49%:%
%:%125=49%:%
%:%126=49%:%
%:%131=49%:%
%:%134=50%:%
%:%135=51%:%
%:%136=51%:%
%:%137=52%:%
%:%140=53%:%
%:%144=53%:%
%:%145=53%:%
%:%150=53%:%
%:%153=54%:%
%:%154=55%:%
%:%155=55%:%
%:%156=56%:%
%:%159=57%:%
%:%163=57%:%
%:%164=57%:%
%:%169=57%:%
%:%172=58%:%
%:%173=59%:%
%:%174=59%:%
%:%175=60%:%
%:%178=61%:%
%:%182=61%:%
%:%183=61%:%
%:%188=61%:%
%:%191=62%:%
%:%192=63%:%
%:%193=63%:%
%:%194=64%:%
%:%197=65%:%
%:%201=65%:%
%:%202=65%:%
%:%207=65%:%
%:%210=66%:%
%:%211=67%:%
%:%212=67%:%
%:%213=68%:%
%:%216=69%:%
%:%220=69%:%
%:%221=69%:%
%:%226=69%:%
%:%229=70%:%
%:%230=71%:%
%:%231=71%:%
%:%232=72%:%
%:%235=73%:%
%:%239=73%:%
%:%240=73%:%
%:%245=73%:%
%:%248=74%:%
%:%249=75%:%
%:%250=76%:%
%:%251=76%:%
%:%252=77%:%
%:%255=78%:%
%:%259=78%:%
%:%260=78%:%
%:%265=78%:%
%:%268=79%:%
%:%269=80%:%
%:%270=80%:%
%:%271=81%:%
%:%274=82%:%
%:%278=82%:%
%:%279=82%:%
%:%293=85%:%
%:%305=88%:%
%:%306=89%:%
%:%307=90%:%
%:%308=91%:%
%:%310=94%:%
%:%311=94%:%
%:%312=95%:%
%:%313=96%:%
%:%315=99%:%
%:%316=100%:%
%:%317=101%:%
%:%319=104%:%
%:%320=104%:%
%:%321=105%:%
%:%322=106%:%
%:%323=107%:%
%:%324=108%:%
%:%325=109%:%
%:%326=110%:%
%:%327=110%:%
%:%328=111%:%
%:%331=112%:%
%:%335=112%:%
%:%336=112%:%
%:%337=113%:%
%:%338=113%:%
%:%343=113%:%
%:%346=114%:%
%:%347=115%:%
%:%348=115%:%
%:%349=116%:%
%:%352=117%:%
%:%356=117%:%
%:%357=117%:%
%:%358=118%:%
%:%359=118%:%
%:%364=118%:%
%:%367=119%:%
%:%368=120%:%
%:%369=120%:%
%:%370=121%:%
%:%373=122%:%
%:%377=122%:%
%:%378=122%:%
%:%379=122%:%
%:%384=122%:%
%:%387=123%:%
%:%388=124%:%
%:%389=124%:%
%:%390=125%:%
%:%393=126%:%
%:%397=126%:%
%:%398=126%:%
%:%403=126%:%
%:%406=127%:%
%:%407=128%:%
%:%408=128%:%
%:%409=129%:%
%:%412=130%:%
%:%416=130%:%
%:%417=130%:%
%:%422=130%:%
%:%425=131%:%
%:%426=132%:%
%:%427=132%:%
%:%428=133%:%
%:%431=134%:%
%:%435=134%:%
%:%436=134%:%
%:%437=135%:%
%:%438=135%:%
%:%439=136%:%
%:%440=136%:%
%:%441=136%:%
%:%446=136%:%
%:%451=137%:%
%:%456=138%:%

%
\begin{isabellebody}%
\setisabellecontext{Non{\isacharunderscore}{\kern0pt}Boolean{\isacharunderscore}{\kern0pt}Gray}%
%
\isadelimdocument
%
\endisadelimdocument
%
\isatagdocument
%
\isamarkupsection{A non-Boolean Gray code%
}
\isamarkuptrue%
%
\endisatagdocument
{\isafolddocument}%
%
\isadelimdocument
%
\endisadelimdocument
%
\isadelimtheory
%
\endisadelimtheory
%
\isatagtheory
\isacommand{theory}\isamarkupfalse%
\ Non{\isacharunderscore}{\kern0pt}Boolean{\isacharunderscore}{\kern0pt}Gray\isanewline
\ \ \isakeyword{imports}\ Code{\isacharunderscore}{\kern0pt}Word{\isacharunderscore}{\kern0pt}Dist\isanewline
\isakeyword{begin}%
\endisatagtheory
{\isafoldtheory}%
%
\isadelimtheory
%
\endisadelimtheory
%
\begin{isamarkuptext}%
The function presented below transforms a code word into a gray code
  and the corresponding decode function is exactly its inverse.
The key idea is to shift down a digit by the prefix sum of gray digits.
A crucial property is the behavior of this prefix sum under increment
  as stated below.%
\end{isamarkuptext}\isamarkuptrue%
\isacommand{fun}\isamarkupfalse%
\ to{\isacharunderscore}{\kern0pt}gray\ {\isacharcolon}{\kern0pt}{\isacharcolon}{\kern0pt}\ {\isachardoublequoteopen}base\ {\isasymRightarrow}\ word\ {\isasymRightarrow}\ word{\isachardoublequoteclose}\ \isakeyword{where}\isanewline
\ \ {\isachardoublequoteopen}to{\isacharunderscore}{\kern0pt}gray\ {\isacharunderscore}{\kern0pt}\ {\isacharbrackleft}{\kern0pt}{\isacharbrackright}{\kern0pt}\ {\isacharequal}{\kern0pt}\ {\isacharbrackleft}{\kern0pt}{\isacharbrackright}{\kern0pt}{\isachardoublequoteclose}\isanewline
{\isacharbar}{\kern0pt}\ {\isachardoublequoteopen}to{\isacharunderscore}{\kern0pt}gray\ b\ {\isacharparenleft}{\kern0pt}a{\isacharhash}{\kern0pt}v{\isacharparenright}{\kern0pt}\ {\isacharequal}{\kern0pt}\ {\isacharparenleft}{\kern0pt}let\ g{\isacharequal}{\kern0pt}to{\isacharunderscore}{\kern0pt}gray\ b\ v\ in\ dist{\isadigit{1}}\ b\ {\isacharparenleft}{\kern0pt}sum{\isacharunderscore}{\kern0pt}list\ g\ mod\ b{\isacharparenright}{\kern0pt}\ a{\isacharhash}{\kern0pt}g{\isacharparenright}{\kern0pt}{\isachardoublequoteclose}\isanewline
\isanewline
\isacommand{fun}\isamarkupfalse%
\ decode\ {\isacharcolon}{\kern0pt}{\isacharcolon}{\kern0pt}\ {\isachardoublequoteopen}base\ {\isasymRightarrow}\ word\ {\isasymRightarrow}\ word{\isachardoublequoteclose}\ \isakeyword{where}\isanewline
\ \ {\isachardoublequoteopen}decode\ {\isacharunderscore}{\kern0pt}\ {\isacharbrackleft}{\kern0pt}{\isacharbrackright}{\kern0pt}\ {\isacharequal}{\kern0pt}\ {\isacharbrackleft}{\kern0pt}{\isacharbrackright}{\kern0pt}{\isachardoublequoteclose}\isanewline
{\isacharbar}{\kern0pt}\ {\isachardoublequoteopen}decode\ b\ {\isacharparenleft}{\kern0pt}g{\isacharhash}{\kern0pt}c{\isacharparenright}{\kern0pt}\ {\isacharequal}{\kern0pt}\ {\isacharparenleft}{\kern0pt}g{\isacharplus}{\kern0pt}sum{\isacharunderscore}{\kern0pt}list\ c\ mod\ b{\isacharparenright}{\kern0pt}\ mod\ b{\isacharhash}{\kern0pt}decode\ b\ c{\isachardoublequoteclose}%
\isadelimdocument
%
\endisadelimdocument
%
\isatagdocument
%
\isamarkupsubsection{The Correctness Proof%
}
\isamarkuptrue%
%
\endisatagdocument
{\isafolddocument}%
%
\isadelimdocument
%
\endisadelimdocument
%
\begin{isamarkuptext}%
The proof of all properties that are necessary for a gray code
  is presented below. Also, some auxiliary lemmas are required:%
\end{isamarkuptext}\isamarkuptrue%
\isacommand{lemma}\isamarkupfalse%
\ length{\isacharunderscore}{\kern0pt}gray{\isacharcolon}{\kern0pt}\isanewline
\ \ {\isachardoublequoteopen}length\ {\isacharparenleft}{\kern0pt}to{\isacharunderscore}{\kern0pt}gray\ b\ w{\isacharparenright}{\kern0pt}\ {\isacharequal}{\kern0pt}\ length\ w{\isachardoublequoteclose}\isanewline
%
\isadelimproof
\ \ %
\endisadelimproof
%
\isatagproof
\isacommand{apply}\isamarkupfalse%
\ {\isacharparenleft}{\kern0pt}induction\ w{\isacharparenright}{\kern0pt}\isanewline
\ \ \isacommand{by}\isamarkupfalse%
\ {\isacharparenleft}{\kern0pt}auto\ simp\ add{\isacharcolon}{\kern0pt}\ Let{\isacharunderscore}{\kern0pt}def{\isacharparenright}{\kern0pt}%
\endisatagproof
{\isafoldproof}%
%
\isadelimproof
\isanewline
%
\endisadelimproof
\isanewline
\isacommand{lemma}\isamarkupfalse%
\ valid{\isacharunderscore}{\kern0pt}gray{\isacharcolon}{\kern0pt}\isanewline
\ \ {\isachardoublequoteopen}valid\ b\ w\ {\isasymLongrightarrow}\ valid\ b\ {\isacharparenleft}{\kern0pt}to{\isacharunderscore}{\kern0pt}gray\ b\ w{\isacharparenright}{\kern0pt}{\isachardoublequoteclose}\isanewline
%
\isadelimproof
\ \ %
\endisadelimproof
%
\isatagproof
\isacommand{apply}\isamarkupfalse%
\ {\isacharparenleft}{\kern0pt}induction\ w{\isacharparenright}{\kern0pt}\isanewline
\ \ \isacommand{by}\isamarkupfalse%
\ {\isacharparenleft}{\kern0pt}auto\ simp\ add{\isacharcolon}{\kern0pt}\ dist{\isadigit{1}}{\isacharunderscore}{\kern0pt}valid\ Let{\isacharunderscore}{\kern0pt}def{\isacharparenright}{\kern0pt}%
\endisatagproof
{\isafoldproof}%
%
\isadelimproof
%
\endisadelimproof
%
\begin{isamarkuptext}%
The sum of grays is congruent to the value (mod $b$):%
\end{isamarkuptext}\isamarkuptrue%
\isacommand{lemma}\isamarkupfalse%
\ prefix{\isacharunderscore}{\kern0pt}sum{\isacharcolon}{\kern0pt}\isanewline
\ \ {\isachardoublequoteopen}valid\ b\ w\ {\isasymLongrightarrow}\ sum{\isacharunderscore}{\kern0pt}list\ {\isacharparenleft}{\kern0pt}to{\isacharunderscore}{\kern0pt}gray\ b\ w{\isacharparenright}{\kern0pt}\ mod\ b\ {\isacharequal}{\kern0pt}\ val\ b\ w\ mod\ b{\isachardoublequoteclose}\isanewline
%
\isadelimproof
%
\endisadelimproof
%
\isatagproof
\isacommand{proof}\isamarkupfalse%
\ {\isacharparenleft}{\kern0pt}induction\ w{\isacharparenright}{\kern0pt}\isanewline
\ \ \isacommand{case}\isamarkupfalse%
\ Nil\ \isacommand{thus}\isamarkupfalse%
\ {\isacharquery}{\kern0pt}case\ \isacommand{by}\isamarkupfalse%
\ simp\isanewline
\isacommand{next}\isamarkupfalse%
\isanewline
\ \ \isacommand{case}\isamarkupfalse%
\ {\isacharparenleft}{\kern0pt}Cons\ a\ w{\isacharparenright}{\kern0pt}\isanewline
\ \ \isacommand{hence}\isamarkupfalse%
\ IH{\isacharcolon}{\kern0pt}\ {\isachardoublequoteopen}sum{\isacharunderscore}{\kern0pt}list\ {\isacharparenleft}{\kern0pt}to{\isacharunderscore}{\kern0pt}gray\ b\ w{\isacharparenright}{\kern0pt}\ mod\ b\ {\isacharequal}{\kern0pt}\ val\ b\ w\ mod\ b{\isachardoublequoteclose}\ \isacommand{by}\isamarkupfalse%
\ simp\isanewline
\ \ \isacommand{let}\isamarkupfalse%
\ {\isacharquery}{\kern0pt}s\ {\isacharequal}{\kern0pt}\ {\isachardoublequoteopen}sum{\isacharunderscore}{\kern0pt}list\ {\isacharparenleft}{\kern0pt}to{\isacharunderscore}{\kern0pt}gray\ b\ w{\isacharparenright}{\kern0pt}{\isachardoublequoteclose}\isanewline
\ \ \isacommand{let}\isamarkupfalse%
\ {\isacharquery}{\kern0pt}v\ {\isacharequal}{\kern0pt}\ {\isachardoublequoteopen}val\ b\ w\ mod\ b{\isachardoublequoteclose}\isanewline
\ \ \isacommand{have}\isamarkupfalse%
\ {\isachardoublequoteopen}{\isacharparenleft}{\kern0pt}dist{\isadigit{1}}\ b\ {\isacharquery}{\kern0pt}v\ a\ {\isacharplus}{\kern0pt}\ {\isacharquery}{\kern0pt}s{\isacharparenright}{\kern0pt}\ mod\ b\ {\isacharequal}{\kern0pt}\ {\isacharparenleft}{\kern0pt}dist{\isadigit{1}}\ b\ {\isacharquery}{\kern0pt}v\ a\ {\isacharplus}{\kern0pt}\ {\isacharquery}{\kern0pt}s\ mod\ b{\isacharparenright}{\kern0pt}\ mod\ b{\isachardoublequoteclose}\ \isacommand{by}\isamarkupfalse%
\ presburger\isanewline
\ \ \isacommand{also}\isamarkupfalse%
\ \isacommand{have}\isamarkupfalse%
\ {\isachardoublequoteopen}{\isachardot}{\kern0pt}{\isachardot}{\kern0pt}{\isachardot}{\kern0pt}\ {\isacharequal}{\kern0pt}\ {\isacharparenleft}{\kern0pt}dist{\isadigit{1}}\ b\ {\isacharquery}{\kern0pt}v\ a\ {\isacharplus}{\kern0pt}\ {\isacharquery}{\kern0pt}v{\isacharparenright}{\kern0pt}\ mod\ b{\isachardoublequoteclose}\ \isacommand{using}\isamarkupfalse%
\ IH\ \isacommand{by}\isamarkupfalse%
\ argo\isanewline
\ \ \isacommand{also}\isamarkupfalse%
\ \isacommand{have}\isamarkupfalse%
\ {\isachardoublequoteopen}{\isachardot}{\kern0pt}{\isachardot}{\kern0pt}{\isachardot}{\kern0pt}\ {\isacharequal}{\kern0pt}\ a{\isachardoublequoteclose}\ \isacommand{using}\isamarkupfalse%
\ Cons{\isachardot}{\kern0pt}prems\ dist{\isadigit{1}}{\isacharunderscore}{\kern0pt}elim{\isacharunderscore}{\kern0pt}{\isadigit{1}}\ \isacommand{by}\isamarkupfalse%
\ simp\isanewline
\ \ \isacommand{finally}\isamarkupfalse%
\ \isacommand{show}\isamarkupfalse%
\ {\isacharquery}{\kern0pt}case\ \isacommand{using}\isamarkupfalse%
\ Cons\ \isacommand{by}\isamarkupfalse%
\ auto\isanewline
\isacommand{qed}\isamarkupfalse%
%
\endisatagproof
{\isafoldproof}%
%
\isadelimproof
\isanewline
%
\endisadelimproof
\isanewline
\isacommand{lemma}\isamarkupfalse%
\ decode{\isacharunderscore}{\kern0pt}correct{\isacharcolon}{\kern0pt}\isanewline
\ \ {\isachardoublequoteopen}valid\ b\ w\ {\isasymLongrightarrow}\ decode\ b\ {\isacharparenleft}{\kern0pt}to{\isacharunderscore}{\kern0pt}gray\ b\ w{\isacharparenright}{\kern0pt}\ {\isacharequal}{\kern0pt}\ w{\isachardoublequoteclose}\isanewline
%
\isadelimproof
\ \ %
\endisadelimproof
%
\isatagproof
\isacommand{apply}\isamarkupfalse%
\ {\isacharparenleft}{\kern0pt}induction\ w{\isacharparenright}{\kern0pt}\isanewline
\ \ \isacommand{by}\isamarkupfalse%
\ {\isacharparenleft}{\kern0pt}auto\ simp\ add{\isacharcolon}{\kern0pt}\ Let{\isacharunderscore}{\kern0pt}def\ dist{\isadigit{1}}{\isacharunderscore}{\kern0pt}elim{\isacharunderscore}{\kern0pt}{\isadigit{1}}{\isacharparenright}{\kern0pt}%
\endisatagproof
{\isafoldproof}%
%
\isadelimproof
%
\endisadelimproof
%
\begin{isamarkuptext}%
The following theorem states that the transformation to gray
  is an encoding of the valid code words:%
\end{isamarkuptext}\isamarkuptrue%
\isacommand{theorem}\isamarkupfalse%
\ gray{\isacharunderscore}{\kern0pt}encoding{\isacharcolon}{\kern0pt}\isanewline
\ \ {\isachardoublequoteopen}inj{\isacharunderscore}{\kern0pt}on\ {\isacharparenleft}{\kern0pt}to{\isacharunderscore}{\kern0pt}gray\ b{\isacharparenright}{\kern0pt}\ {\isacharbraceleft}{\kern0pt}w{\isachardot}{\kern0pt}\ valid\ b\ w{\isacharbraceright}{\kern0pt}{\isachardoublequoteclose}\isanewline
%
\isadelimproof
%
\endisadelimproof
%
\isatagproof
\isacommand{proof}\isamarkupfalse%
\ {\isacharparenleft}{\kern0pt}rule\ inj{\isacharunderscore}{\kern0pt}on{\isacharunderscore}{\kern0pt}inverseI{\isacharparenright}{\kern0pt}\isanewline
\ \ \isacommand{fix}\isamarkupfalse%
\ w\ {\isacharcolon}{\kern0pt}{\isacharcolon}{\kern0pt}\ word\isanewline
\ \ \isacommand{assume}\isamarkupfalse%
\ {\isachardoublequoteopen}w\ {\isasymin}\ {\isacharbraceleft}{\kern0pt}w{\isachardot}{\kern0pt}\ valid\ b\ w{\isacharbraceright}{\kern0pt}{\isachardoublequoteclose}\isanewline
\ \ \isacommand{hence}\isamarkupfalse%
\ {\isachardoublequoteopen}valid\ b\ w{\isachardoublequoteclose}\ \isacommand{by}\isamarkupfalse%
\ blast\isanewline
\ \ \isacommand{thus}\isamarkupfalse%
\ {\isachardoublequoteopen}decode\ b\ {\isacharparenleft}{\kern0pt}to{\isacharunderscore}{\kern0pt}gray\ b\ w{\isacharparenright}{\kern0pt}\ {\isacharequal}{\kern0pt}\ w{\isachardoublequoteclose}\ \isacommand{using}\isamarkupfalse%
\ decode{\isacharunderscore}{\kern0pt}correct\ \isacommand{by}\isamarkupfalse%
\ simp\isanewline
\isacommand{qed}\isamarkupfalse%
%
\endisatagproof
{\isafoldproof}%
%
\isadelimproof
\isanewline
%
\endisadelimproof
\isanewline
\isacommand{lemma}\isamarkupfalse%
\ mod{\isacharunderscore}{\kern0pt}mod{\isacharunderscore}{\kern0pt}aux{\isacharcolon}{\kern0pt}\ {\isachardoublequoteopen}{\isadigit{1}}\ {\isasymle}\ k\ {\isasymLongrightarrow}\ {\isacharparenleft}{\kern0pt}a{\isacharcolon}{\kern0pt}{\isacharcolon}{\kern0pt}nat{\isacharparenright}{\kern0pt}\ mod\ b{\isacharcircum}{\kern0pt}k\ mod\ b\ {\isacharequal}{\kern0pt}\ a\ mod\ b{\isachardoublequoteclose}\isanewline
%
\isadelimproof
\ \ %
\endisadelimproof
%
\isatagproof
\isacommand{by}\isamarkupfalse%
\ {\isacharparenleft}{\kern0pt}simp\ add{\isacharcolon}{\kern0pt}\ mod{\isacharunderscore}{\kern0pt}mod{\isacharunderscore}{\kern0pt}cancel{\isacharparenright}{\kern0pt}%
\endisatagproof
{\isafoldproof}%
%
\isadelimproof
\isanewline
%
\endisadelimproof
\isanewline
\isacommand{lemma}\isamarkupfalse%
\ gray{\isacharunderscore}{\kern0pt}dist{\isacharcolon}{\kern0pt}\isanewline
\ \ {\isachardoublequoteopen}valid\ b\ w\ {\isasymLongrightarrow}\ dist\ b\ {\isacharparenleft}{\kern0pt}to{\isacharunderscore}{\kern0pt}gray\ b\ w{\isacharparenright}{\kern0pt}\ {\isacharparenleft}{\kern0pt}to{\isacharunderscore}{\kern0pt}gray\ b\ {\isacharparenleft}{\kern0pt}inc\ b\ w{\isacharparenright}{\kern0pt}{\isacharparenright}{\kern0pt}\ {\isasymle}\ {\isadigit{1}}{\isachardoublequoteclose}\isanewline
%
\isadelimproof
%
\endisadelimproof
%
\isatagproof
\isacommand{proof}\isamarkupfalse%
\ {\isacharparenleft}{\kern0pt}induction\ w{\isacharparenright}{\kern0pt}\isanewline
\ \ \isacommand{case}\isamarkupfalse%
\ Nil\ \isacommand{thus}\isamarkupfalse%
\ {\isacharquery}{\kern0pt}case\ \isacommand{by}\isamarkupfalse%
\ simp\isanewline
\isacommand{next}\isamarkupfalse%
\isanewline
\ \ \isacommand{case}\isamarkupfalse%
\ {\isacharparenleft}{\kern0pt}Cons\ a\ w{\isacharparenright}{\kern0pt}\isanewline
\ \ \isacommand{have}\isamarkupfalse%
\ {\isachardoublequoteopen}valid\ b\ w{\isachardoublequoteclose}\ \isacommand{using}\isamarkupfalse%
\ Cons{\isachardot}{\kern0pt}prems\ \isacommand{by}\isamarkupfalse%
\ simp\isanewline
\ \ \isacommand{hence}\isamarkupfalse%
\ {\isachardoublequoteopen}{\isadigit{2}}\ {\isasymle}\ b{\isachardoublequoteclose}\ \isacommand{using}\isamarkupfalse%
\ valid{\isacharunderscore}{\kern0pt}base\ \isacommand{by}\isamarkupfalse%
\ auto\isanewline
\ \ \isacommand{hence}\isamarkupfalse%
\ {\isachardoublequoteopen}{\isadigit{0}}\ {\isacharless}{\kern0pt}\ b{\isachardoublequoteclose}\ \isacommand{by}\isamarkupfalse%
\ simp\isanewline
\ \ \isacommand{have}\isamarkupfalse%
\ IH{\isacharcolon}{\kern0pt}\ {\isachardoublequoteopen}dist\ b\ {\isacharparenleft}{\kern0pt}to{\isacharunderscore}{\kern0pt}gray\ b\ w{\isacharparenright}{\kern0pt}\ {\isacharparenleft}{\kern0pt}to{\isacharunderscore}{\kern0pt}gray\ b\ {\isacharparenleft}{\kern0pt}inc\ b\ w{\isacharparenright}{\kern0pt}{\isacharparenright}{\kern0pt}\ {\isasymle}\ {\isadigit{1}}{\isachardoublequoteclose}\isanewline
\ \ \ \ \isacommand{using}\isamarkupfalse%
\ {\isacartoucheopen}valid\ b\ w{\isacartoucheclose}\ Cons{\isachardot}{\kern0pt}IH\ \isacommand{by}\isamarkupfalse%
\ blast\isanewline
\ \ \isacommand{have}\isamarkupfalse%
\ {\isachardoublequoteopen}a\ {\isacharless}{\kern0pt}\ b{\isachardoublequoteclose}\ \isacommand{using}\isamarkupfalse%
\ Cons{\isachardot}{\kern0pt}prems\ \isacommand{by}\isamarkupfalse%
\ simp\isanewline
\ \ \isacommand{show}\isamarkupfalse%
\ {\isacharquery}{\kern0pt}case\isanewline
\ \ \isacommand{proof}\isamarkupfalse%
\ {\isacharparenleft}{\kern0pt}cases\ w{\isacharparenright}{\kern0pt}\isanewline
\ \ \ \ \isacommand{case}\isamarkupfalse%
\ Nil\ \isacommand{thus}\isamarkupfalse%
\ {\isacharquery}{\kern0pt}thesis\isanewline
\ \ \ \ \ \ \isacommand{using}\isamarkupfalse%
\ dist{\isadigit{1}}{\isacharunderscore}{\kern0pt}distr\ dist{\isadigit{1}}{\isacharunderscore}{\kern0pt}Suc\ {\isacartoucheopen}a\ {\isacharless}{\kern0pt}\ b{\isacartoucheclose}\ {\isacartoucheopen}{\isadigit{2}}\ {\isasymle}\ b{\isacartoucheclose}\ \isacommand{by}\isamarkupfalse%
\ simp\isanewline
\ \ \isacommand{next}\isamarkupfalse%
\isanewline
\ \ \ \ \isacommand{case}\isamarkupfalse%
\ {\isacharparenleft}{\kern0pt}Cons\ a{\isacharprime}{\kern0pt}\ ds{\isacharprime}{\kern0pt}{\isacharparenright}{\kern0pt}\isanewline
\ \ \ \ \isacommand{hence}\isamarkupfalse%
\ {\isachardoublequoteopen}{\isadigit{1}}{\isasymle}length{\isacharparenleft}{\kern0pt}w{\isacharparenright}{\kern0pt}{\isachardoublequoteclose}\ \isacommand{by}\isamarkupfalse%
\ simp\isanewline
\ \ \ \ \isacommand{let}\isamarkupfalse%
\ {\isacharquery}{\kern0pt}a\ {\isacharequal}{\kern0pt}\ {\isachardoublequoteopen}if\ Suc\ a{\isasymnoteq}b\ then\ w\ else\ inc\ b\ w{\isachardoublequoteclose}\isanewline
\ \ \ \ \isacommand{let}\isamarkupfalse%
\ {\isacharquery}{\kern0pt}g\ {\isacharequal}{\kern0pt}\ {\isachardoublequoteopen}sum{\isacharunderscore}{\kern0pt}list\ {\isacharparenleft}{\kern0pt}to{\isacharunderscore}{\kern0pt}gray\ b\ w{\isacharparenright}{\kern0pt}\ mod\ b{\isachardoublequoteclose}\isanewline
\ \ \ \ \isacommand{let}\isamarkupfalse%
\ {\isacharquery}{\kern0pt}h\ {\isacharequal}{\kern0pt}\ {\isachardoublequoteopen}sum{\isacharunderscore}{\kern0pt}list\ {\isacharparenleft}{\kern0pt}to{\isacharunderscore}{\kern0pt}gray\ b\ {\isacharquery}{\kern0pt}a{\isacharparenright}{\kern0pt}\ mod\ b{\isachardoublequoteclose}\isanewline
\ \ \ \ \isacommand{let}\isamarkupfalse%
\ {\isacharquery}{\kern0pt}v\ {\isacharequal}{\kern0pt}\ {\isachardoublequoteopen}val\ b\ w\ mod\ b{\isachardoublequoteclose}\isanewline
\ \ \ \ \isacommand{let}\isamarkupfalse%
\ {\isacharquery}{\kern0pt}u\ {\isacharequal}{\kern0pt}\ {\isachardoublequoteopen}val\ b\ {\isacharquery}{\kern0pt}a\ mod\ b{\isachardoublequoteclose}\isanewline
\ \ \ \ \isacommand{let}\isamarkupfalse%
\ {\isacharquery}{\kern0pt}l\ {\isacharequal}{\kern0pt}\ {\isachardoublequoteopen}dist\ b\ {\isacharparenleft}{\kern0pt}to{\isacharunderscore}{\kern0pt}gray\ b\ {\isacharparenleft}{\kern0pt}a{\isacharhash}{\kern0pt}w{\isacharparenright}{\kern0pt}{\isacharparenright}{\kern0pt}\ {\isacharparenleft}{\kern0pt}to{\isacharunderscore}{\kern0pt}gray\ b\ {\isacharparenleft}{\kern0pt}inc\ b\ {\isacharparenleft}{\kern0pt}a{\isacharhash}{\kern0pt}w{\isacharparenright}{\kern0pt}{\isacharparenright}{\kern0pt}{\isacharparenright}{\kern0pt}{\isachardoublequoteclose}\isanewline
\ \ \ \ \isacommand{have}\isamarkupfalse%
\ {\isachardoublequoteopen}valid\ b\ {\isacharquery}{\kern0pt}a{\isachardoublequoteclose}\ \isacommand{using}\isamarkupfalse%
\ {\isacartoucheopen}valid\ b\ w{\isacartoucheclose}\ valid{\isacharunderscore}{\kern0pt}inc\ \isacommand{by}\isamarkupfalse%
\ simp\isanewline
\ \ \ \ \isacommand{have}\isamarkupfalse%
\ {\isachardoublequoteopen}{\isacharquery}{\kern0pt}l\ {\isacharequal}{\kern0pt}\ dist{\isadigit{1}}\ b\ {\isacharparenleft}{\kern0pt}dist{\isadigit{1}}\ b\ {\isacharquery}{\kern0pt}g\ a{\isacharparenright}{\kern0pt}\ {\isacharparenleft}{\kern0pt}dist{\isadigit{1}}\ b\ {\isacharquery}{\kern0pt}h\ {\isacharparenleft}{\kern0pt}Suc\ a\ mod\ b{\isacharparenright}{\kern0pt}{\isacharparenright}{\kern0pt}\isanewline
\ \ \ \ \ \ \ \ \ \ \ \ \ {\isacharplus}{\kern0pt}\ dist\ b\ {\isacharparenleft}{\kern0pt}to{\isacharunderscore}{\kern0pt}gray\ b\ w{\isacharparenright}{\kern0pt}\ {\isacharparenleft}{\kern0pt}to{\isacharunderscore}{\kern0pt}gray\ b\ {\isacharquery}{\kern0pt}a{\isacharparenright}{\kern0pt}{\isachardoublequoteclose}\isanewline
\ \ \ \ \ \ \isacommand{by}\isamarkupfalse%
\ {\isacharparenleft}{\kern0pt}metis\ Encoding{\isacharunderscore}{\kern0pt}Nat{\isachardot}{\kern0pt}inc{\isachardot}{\kern0pt}simps{\isacharparenleft}{\kern0pt}{\isadigit{2}}{\isacharparenright}{\kern0pt}\ dist{\isachardot}{\kern0pt}simps{\isacharparenleft}{\kern0pt}{\isadigit{4}}{\isacharparenright}{\kern0pt}\ to{\isacharunderscore}{\kern0pt}gray{\isachardot}{\kern0pt}simps{\isacharparenleft}{\kern0pt}{\isadigit{2}}{\isacharparenright}{\kern0pt}{\isacharparenright}{\kern0pt}\isanewline
\ \ \ \ \isacommand{also}\isamarkupfalse%
\ \isacommand{have}\isamarkupfalse%
\ {\isachardoublequoteopen}{\isachardot}{\kern0pt}{\isachardot}{\kern0pt}{\isachardot}{\kern0pt}\ {\isacharequal}{\kern0pt}\ Suc\ {\isacharparenleft}{\kern0pt}dist{\isadigit{1}}\ b\ {\isacharparenleft}{\kern0pt}dist{\isadigit{1}}\ b\ {\isacharquery}{\kern0pt}g\ a{\isacharparenright}{\kern0pt}\ {\isacharparenleft}{\kern0pt}dist{\isadigit{1}}\ b\ {\isacharquery}{\kern0pt}h\ a{\isacharparenright}{\kern0pt}{\isacharparenright}{\kern0pt}\ mod\ b\isanewline
\ \ \ \ \ \ \ \ \ \ \ \ \ {\isacharplus}{\kern0pt}\ dist\ b\ {\isacharparenleft}{\kern0pt}to{\isacharunderscore}{\kern0pt}gray\ b\ w{\isacharparenright}{\kern0pt}\ {\isacharparenleft}{\kern0pt}to{\isacharunderscore}{\kern0pt}gray\ b\ {\isacharquery}{\kern0pt}a{\isacharparenright}{\kern0pt}{\isachardoublequoteclose}\isanewline
\ \ \ \ \ \ \isacommand{using}\isamarkupfalse%
\ {\isacartoucheopen}a\ {\isacharless}{\kern0pt}\ b{\isacartoucheclose}\ dist{\isadigit{1}}{\isacharunderscore}{\kern0pt}mod{\isacharunderscore}{\kern0pt}Suc\ dist{\isadigit{1}}{\isacharunderscore}{\kern0pt}valid\ \isacommand{by}\isamarkupfalse%
\ simp\isanewline
\ \ \ \ \isacommand{also}\isamarkupfalse%
\ \isacommand{have}\isamarkupfalse%
\ {\isachardoublequoteopen}{\isachardot}{\kern0pt}{\isachardot}{\kern0pt}{\isachardot}{\kern0pt}\ {\isacharequal}{\kern0pt}\ Suc\ {\isacharparenleft}{\kern0pt}dist{\isadigit{1}}\ b\ {\isacharquery}{\kern0pt}h\ {\isacharquery}{\kern0pt}g{\isacharparenright}{\kern0pt}\ mod\ b\isanewline
\ \ \ \ \ \ \ \ \ \ \ \ \ {\isacharplus}{\kern0pt}\ dist\ b\ {\isacharparenleft}{\kern0pt}to{\isacharunderscore}{\kern0pt}gray\ b\ w{\isacharparenright}{\kern0pt}\ {\isacharparenleft}{\kern0pt}to{\isacharunderscore}{\kern0pt}gray\ b\ {\isacharquery}{\kern0pt}a{\isacharparenright}{\kern0pt}{\isachardoublequoteclose}\isanewline
\ \ \ \ \ \ \isacommand{using}\isamarkupfalse%
\ {\isacartoucheopen}a\ {\isacharless}{\kern0pt}\ b{\isacartoucheclose}\ dist{\isadigit{1}}{\isacharunderscore}{\kern0pt}distr{\isadigit{2}}\ \isacommand{by}\isamarkupfalse%
\ simp\isanewline
\ \ \ \ \isacommand{also}\isamarkupfalse%
\ \isacommand{have}\isamarkupfalse%
\ {\isachardoublequoteopen}{\isachardot}{\kern0pt}{\isachardot}{\kern0pt}{\isachardot}{\kern0pt}\ {\isacharequal}{\kern0pt}\ Suc\ {\isacharparenleft}{\kern0pt}dist{\isadigit{1}}\ b\ {\isacharquery}{\kern0pt}h\ {\isacharquery}{\kern0pt}v{\isacharparenright}{\kern0pt}\ mod\ b\isanewline
\ \ \ \ \ \ \ \ \ \ \ \ \ {\isacharplus}{\kern0pt}\ dist\ b\ {\isacharparenleft}{\kern0pt}to{\isacharunderscore}{\kern0pt}gray\ b\ w{\isacharparenright}{\kern0pt}\ {\isacharparenleft}{\kern0pt}to{\isacharunderscore}{\kern0pt}gray\ b\ {\isacharquery}{\kern0pt}a{\isacharparenright}{\kern0pt}{\isachardoublequoteclose}\isanewline
\ \ \ \ \ \ \isacommand{using}\isamarkupfalse%
\ {\isacartoucheopen}valid\ b\ w{\isacartoucheclose}\ prefix{\isacharunderscore}{\kern0pt}sum\ \isacommand{by}\isamarkupfalse%
\ simp\isanewline
\ \ \ \ \isacommand{also}\isamarkupfalse%
\ \isacommand{have}\isamarkupfalse%
\ {\isachardoublequoteopen}{\isachardot}{\kern0pt}{\isachardot}{\kern0pt}{\isachardot}{\kern0pt}\ {\isacharequal}{\kern0pt}\ Suc\ {\isacharparenleft}{\kern0pt}dist{\isadigit{1}}\ b\ {\isacharquery}{\kern0pt}u\ {\isacharquery}{\kern0pt}v{\isacharparenright}{\kern0pt}\ mod\ b\isanewline
\ \ \ \ \ \ \ \ \ \ \ \ \ {\isacharplus}{\kern0pt}\ dist\ b\ {\isacharparenleft}{\kern0pt}to{\isacharunderscore}{\kern0pt}gray\ b\ w{\isacharparenright}{\kern0pt}\ {\isacharparenleft}{\kern0pt}to{\isacharunderscore}{\kern0pt}gray\ b\ {\isacharquery}{\kern0pt}a{\isacharparenright}{\kern0pt}{\isachardoublequoteclose}\isanewline
\ \ \ \ \ \ \isacommand{using}\isamarkupfalse%
\ {\isacartoucheopen}valid\ b\ {\isacharquery}{\kern0pt}a{\isacartoucheclose}\ prefix{\isacharunderscore}{\kern0pt}sum\ \isacommand{by}\isamarkupfalse%
\ simp\isanewline
\ \ \ \ \isacommand{also}\isamarkupfalse%
\ \isacommand{have}\isamarkupfalse%
\ {\isachardoublequoteopen}{\isachardot}{\kern0pt}{\isachardot}{\kern0pt}{\isachardot}{\kern0pt}\ {\isacharequal}{\kern0pt}\ {\isacharparenleft}{\kern0pt}\isanewline
\ \ \ \ \ \ \ \ if\ Suc\ a\ {\isasymnoteq}\ b\ then\ Suc\ {\isadigit{0}}\ mod\ b\isanewline
\ \ \ \ \ \ \ \ else\ Suc\ {\isacharparenleft}{\kern0pt}dist{\isadigit{1}}\ b\ {\isacharparenleft}{\kern0pt}val\ b\ {\isacharparenleft}{\kern0pt}inc\ b\ w{\isacharparenright}{\kern0pt}\ mod\ b{\isacharparenright}{\kern0pt}\ {\isacharquery}{\kern0pt}v{\isacharparenright}{\kern0pt}\ mod\ b\isanewline
\ \ \ \ \ \ \ \ \ \ \ \ \ {\isacharplus}{\kern0pt}\ dist\ b\ {\isacharparenleft}{\kern0pt}to{\isacharunderscore}{\kern0pt}gray\ b\ w{\isacharparenright}{\kern0pt}\ {\isacharparenleft}{\kern0pt}to{\isacharunderscore}{\kern0pt}gray\ b\ {\isacharparenleft}{\kern0pt}inc\ b\ w{\isacharparenright}{\kern0pt}{\isacharparenright}{\kern0pt}{\isacharparenright}{\kern0pt}{\isachardoublequoteclose}\isanewline
\ \ \ \ \ \ \isacommand{using}\isamarkupfalse%
\ dist{\isacharunderscore}{\kern0pt}{\isadigit{0}}\ dist{\isadigit{1}}{\isacharunderscore}{\kern0pt}{\isadigit{0}}\ \isacommand{by}\isamarkupfalse%
\ simp\isanewline
\ \ \ \ \isacommand{also}\isamarkupfalse%
\ \isacommand{have}\isamarkupfalse%
\ {\isachardoublequoteopen}{\isachardot}{\kern0pt}{\isachardot}{\kern0pt}{\isachardot}{\kern0pt}\ {\isacharequal}{\kern0pt}\ {\isacharparenleft}{\kern0pt}\isanewline
\ \ \ \ \ \ \ \ if\ Suc\ a\ {\isasymnoteq}\ b\ then\ Suc\ {\isadigit{0}}\ mod\ b\isanewline
\ \ \ \ \ \ \ \ else\ Suc\ {\isacharparenleft}{\kern0pt}dist{\isadigit{1}}\ b\ {\isacharparenleft}{\kern0pt}Suc\ {\isacharparenleft}{\kern0pt}val\ b\ w{\isacharparenright}{\kern0pt}\ mod\ b{\isacharcircum}{\kern0pt}length{\isacharparenleft}{\kern0pt}w{\isacharparenright}{\kern0pt}\ mod\ b{\isacharparenright}{\kern0pt}\ {\isacharquery}{\kern0pt}v{\isacharparenright}{\kern0pt}\ mod\ b\isanewline
\ \ \ \ \ \ \ \ \ \ \ \ \ {\isacharplus}{\kern0pt}\ dist\ b\ {\isacharparenleft}{\kern0pt}to{\isacharunderscore}{\kern0pt}gray\ b\ w{\isacharparenright}{\kern0pt}\ {\isacharparenleft}{\kern0pt}to{\isacharunderscore}{\kern0pt}gray\ b\ {\isacharparenleft}{\kern0pt}inc\ b\ w{\isacharparenright}{\kern0pt}{\isacharparenright}{\kern0pt}{\isacharparenright}{\kern0pt}{\isachardoublequoteclose}\isanewline
\ \ \ \ \ \ \isacommand{using}\isamarkupfalse%
\ {\isacartoucheopen}valid\ b\ w{\isacartoucheclose}\ valid{\isacharunderscore}{\kern0pt}inc\ val{\isacharunderscore}{\kern0pt}inc\ \isacommand{by}\isamarkupfalse%
\ simp\isanewline
\ \ \ \ \isacommand{also}\isamarkupfalse%
\ \isacommand{have}\isamarkupfalse%
\ {\isachardoublequoteopen}{\isachardot}{\kern0pt}{\isachardot}{\kern0pt}{\isachardot}{\kern0pt}\ {\isacharequal}{\kern0pt}\ {\isacharparenleft}{\kern0pt}\isanewline
\ \ \ \ \ \ \ \ if\ Suc\ a\ {\isasymnoteq}\ b\ then\ Suc\ {\isadigit{0}}\ mod\ b\isanewline
\ \ \ \ \ \ \ \ else\ Suc\ {\isacharparenleft}{\kern0pt}dist{\isadigit{1}}\ b\ {\isacharparenleft}{\kern0pt}Suc\ {\isacharparenleft}{\kern0pt}val\ b\ w{\isacharparenright}{\kern0pt}\ mod\ b{\isacharparenright}{\kern0pt}\ {\isacharquery}{\kern0pt}v{\isacharparenright}{\kern0pt}\ mod\ b\isanewline
\ \ \ \ \ \ \ \ \ \ \ \ \ {\isacharplus}{\kern0pt}\ dist\ b\ {\isacharparenleft}{\kern0pt}to{\isacharunderscore}{\kern0pt}gray\ b\ w{\isacharparenright}{\kern0pt}\ {\isacharparenleft}{\kern0pt}to{\isacharunderscore}{\kern0pt}gray\ b\ {\isacharparenleft}{\kern0pt}inc\ b\ w{\isacharparenright}{\kern0pt}{\isacharparenright}{\kern0pt}{\isacharparenright}{\kern0pt}{\isachardoublequoteclose}\isanewline
\ \ \ \ \ \ \isacommand{using}\isamarkupfalse%
\ {\isacartoucheopen}{\isadigit{1}}{\isasymle}length{\isacharparenleft}{\kern0pt}w{\isacharparenright}{\kern0pt}{\isacartoucheclose}\ mod{\isacharunderscore}{\kern0pt}mod{\isacharunderscore}{\kern0pt}aux\ \isacommand{by}\isamarkupfalse%
\ simp\isanewline
\ \ \ \ \isacommand{also}\isamarkupfalse%
\ \isacommand{have}\isamarkupfalse%
\ {\isachardoublequoteopen}{\isachardot}{\kern0pt}{\isachardot}{\kern0pt}{\isachardot}{\kern0pt}\ {\isacharequal}{\kern0pt}\ {\isacharparenleft}{\kern0pt}\isanewline
\ \ \ \ \ \ \ \ if\ Suc\ a\ {\isasymnoteq}\ b\ then\ Suc\ {\isadigit{0}}\ mod\ b\isanewline
\ \ \ \ \ \ \ \ else\ dist{\isadigit{1}}\ b\ {\isacharparenleft}{\kern0pt}Suc\ {\isacharparenleft}{\kern0pt}val\ b\ w{\isacharparenright}{\kern0pt}\ mod\ b{\isacharparenright}{\kern0pt}\ {\isacharparenleft}{\kern0pt}Suc\ {\isacharquery}{\kern0pt}v\ mod\ b{\isacharparenright}{\kern0pt}\isanewline
\ \ \ \ \ \ \ \ \ \ \ \ \ {\isacharplus}{\kern0pt}\ dist\ b\ {\isacharparenleft}{\kern0pt}to{\isacharunderscore}{\kern0pt}gray\ b\ w{\isacharparenright}{\kern0pt}\ {\isacharparenleft}{\kern0pt}to{\isacharunderscore}{\kern0pt}gray\ b\ {\isacharparenleft}{\kern0pt}inc\ b\ w{\isacharparenright}{\kern0pt}{\isacharparenright}{\kern0pt}{\isacharparenright}{\kern0pt}{\isachardoublequoteclose}\isanewline
\ \ \ \ \ \ \isacommand{using}\isamarkupfalse%
\ dist{\isadigit{1}}{\isacharunderscore}{\kern0pt}mod{\isacharunderscore}{\kern0pt}Suc\ \isacommand{by}\isamarkupfalse%
\ auto\isanewline
\ \ \ \ \isacommand{also}\isamarkupfalse%
\ \isacommand{have}\isamarkupfalse%
\ {\isachardoublequoteopen}{\isachardot}{\kern0pt}{\isachardot}{\kern0pt}{\isachardot}{\kern0pt}\ {\isacharequal}{\kern0pt}\ {\isacharparenleft}{\kern0pt}\isanewline
\ \ \ \ \ \ \ \ if\ Suc\ a\ {\isasymnoteq}\ b\ then\ Suc\ {\isadigit{0}}\ mod\ b\isanewline
\ \ \ \ \ \ \ \ else\ dist{\isadigit{1}}\ b\ {\isacharparenleft}{\kern0pt}Suc\ {\isacharquery}{\kern0pt}v\ mod\ b{\isacharparenright}{\kern0pt}\ {\isacharparenleft}{\kern0pt}Suc\ {\isacharquery}{\kern0pt}v\ mod\ b{\isacharparenright}{\kern0pt}\isanewline
\ \ \ \ \ \ \ \ \ \ \ \ \ {\isacharplus}{\kern0pt}\ dist\ b\ {\isacharparenleft}{\kern0pt}to{\isacharunderscore}{\kern0pt}gray\ b\ w{\isacharparenright}{\kern0pt}\ {\isacharparenleft}{\kern0pt}to{\isacharunderscore}{\kern0pt}gray\ b\ {\isacharparenleft}{\kern0pt}inc\ b\ w{\isacharparenright}{\kern0pt}{\isacharparenright}{\kern0pt}{\isacharparenright}{\kern0pt}{\isachardoublequoteclose}\isanewline
\ \ \ \ \ \ \isacommand{using}\isamarkupfalse%
\ mod{\isacharunderscore}{\kern0pt}Suc{\isacharunderscore}{\kern0pt}eq\ \isacommand{by}\isamarkupfalse%
\ presburger\isanewline
\ \ \ \ \isacommand{also}\isamarkupfalse%
\ \isacommand{have}\isamarkupfalse%
\ {\isachardoublequoteopen}{\isachardot}{\kern0pt}{\isachardot}{\kern0pt}{\isachardot}{\kern0pt}\ {\isacharequal}{\kern0pt}\ {\isacharparenleft}{\kern0pt}\isanewline
\ \ \ \ \ \ \ \ if\ Suc\ a\ {\isasymnoteq}\ b\ then\ Suc\ {\isadigit{0}}\ mod\ b\isanewline
\ \ \ \ \ \ \ \ else\ dist\ b\ {\isacharparenleft}{\kern0pt}to{\isacharunderscore}{\kern0pt}gray\ b\ w{\isacharparenright}{\kern0pt}\ {\isacharparenleft}{\kern0pt}to{\isacharunderscore}{\kern0pt}gray\ b\ {\isacharparenleft}{\kern0pt}inc\ b\ w{\isacharparenright}{\kern0pt}{\isacharparenright}{\kern0pt}{\isacharparenright}{\kern0pt}{\isachardoublequoteclose}\isanewline
\ \ \ \ \ \ \isacommand{using}\isamarkupfalse%
\ dist{\isadigit{1}}{\isacharunderscore}{\kern0pt}{\isadigit{0}}\ \isacommand{by}\isamarkupfalse%
\ simp\isanewline
\ \ \ \ \isacommand{also}\isamarkupfalse%
\ \isacommand{have}\isamarkupfalse%
\ {\isachardoublequoteopen}{\isachardot}{\kern0pt}{\isachardot}{\kern0pt}{\isachardot}{\kern0pt}\ {\isasymle}\ {\isadigit{1}}{\isachardoublequoteclose}\ \isacommand{using}\isamarkupfalse%
\ IH\ \isacommand{by}\isamarkupfalse%
\ simp\isanewline
\ \ \ \ \isacommand{finally}\isamarkupfalse%
\ \isacommand{show}\isamarkupfalse%
\ {\isacharquery}{\kern0pt}thesis\ \isacommand{by}\isamarkupfalse%
\ blast\isanewline
\ \ \isacommand{qed}\isamarkupfalse%
\isanewline
\isacommand{qed}\isamarkupfalse%
%
\endisatagproof
{\isafoldproof}%
%
\isadelimproof
\isanewline
%
\endisadelimproof
\isanewline
\isacommand{lemmas}\isamarkupfalse%
\ gray{\isacharunderscore}{\kern0pt}simps\ {\isacharequal}{\kern0pt}\ decode{\isacharunderscore}{\kern0pt}correct\ dist{\isacharunderscore}{\kern0pt}posd\ inc{\isacharunderscore}{\kern0pt}not{\isacharunderscore}{\kern0pt}eq\ length{\isacharunderscore}{\kern0pt}gray\ length{\isacharunderscore}{\kern0pt}inc\ valid{\isacharunderscore}{\kern0pt}gray\ valid{\isacharunderscore}{\kern0pt}inc\isanewline
\isanewline
\isacommand{lemma}\isamarkupfalse%
\ gray{\isacharunderscore}{\kern0pt}empty{\isacharcolon}{\kern0pt}\isanewline
\ \ {\isachardoublequoteopen}valid\ b\ w\ {\isasymLongrightarrow}\ {\isacharparenleft}{\kern0pt}dist\ b\ {\isacharparenleft}{\kern0pt}to{\isacharunderscore}{\kern0pt}gray\ b\ w{\isacharparenright}{\kern0pt}\ {\isacharparenleft}{\kern0pt}to{\isacharunderscore}{\kern0pt}gray\ b\ {\isacharparenleft}{\kern0pt}inc\ b\ w{\isacharparenright}{\kern0pt}{\isacharparenright}{\kern0pt}\ {\isacharequal}{\kern0pt}\ {\isadigit{0}}{\isacharparenright}{\kern0pt}\ {\isacharequal}{\kern0pt}\ {\isacharparenleft}{\kern0pt}w\ {\isacharequal}{\kern0pt}\ {\isacharbrackleft}{\kern0pt}{\isacharbrackright}{\kern0pt}{\isacharparenright}{\kern0pt}{\isachardoublequoteclose}\isanewline
%
\isadelimproof
\ \ %
\endisadelimproof
%
\isatagproof
\isacommand{by}\isamarkupfalse%
\ {\isacharparenleft}{\kern0pt}metis\ gray{\isacharunderscore}{\kern0pt}simps{\isacharparenright}{\kern0pt}%
\endisatagproof
{\isafoldproof}%
%
\isadelimproof
%
\endisadelimproof
%
\begin{isamarkuptext}%
The central theorem states, that it requires exactly one increment operation
  of one place within the word to go from the gray encoding of a number to
  the gray encoding of its successor.
Note also, that we obtain a cyclic gray code in all cases,
  because the increment operation wraps the last
  number around to zero.
Only the pathological case of an empty word has to be excluded.%
\end{isamarkuptext}\isamarkuptrue%
\isacommand{theorem}\isamarkupfalse%
\ gray{\isacharunderscore}{\kern0pt}correct{\isacharcolon}{\kern0pt}\isanewline
\ \ {\isachardoublequoteopen}{\isasymlbrakk}valid\ b\ w{\isacharsemicolon}{\kern0pt}\ w\ {\isasymnoteq}\ {\isacharbrackleft}{\kern0pt}{\isacharbrackright}{\kern0pt}{\isasymrbrakk}\ {\isasymLongrightarrow}\ dist\ b\ {\isacharparenleft}{\kern0pt}to{\isacharunderscore}{\kern0pt}gray\ b\ w{\isacharparenright}{\kern0pt}\ {\isacharparenleft}{\kern0pt}to{\isacharunderscore}{\kern0pt}gray\ b\ {\isacharparenleft}{\kern0pt}inc\ b\ w{\isacharparenright}{\kern0pt}{\isacharparenright}{\kern0pt}\ {\isacharequal}{\kern0pt}\ {\isadigit{1}}{\isachardoublequoteclose}\isanewline
%
\isadelimproof
%
\endisadelimproof
%
\isatagproof
\isacommand{proof}\isamarkupfalse%
\ {\isacharparenleft}{\kern0pt}rule\ ccontr{\isacharparenright}{\kern0pt}\isanewline
\ \ \isacommand{assume}\isamarkupfalse%
\ a{\isacharcolon}{\kern0pt}\ {\isachardoublequoteopen}dist\ b\ {\isacharparenleft}{\kern0pt}to{\isacharunderscore}{\kern0pt}gray\ b\ w{\isacharparenright}{\kern0pt}\ {\isacharparenleft}{\kern0pt}to{\isacharunderscore}{\kern0pt}gray\ b\ {\isacharparenleft}{\kern0pt}inc\ b\ w{\isacharparenright}{\kern0pt}{\isacharparenright}{\kern0pt}\ {\isasymnoteq}\ {\isadigit{1}}{\isachardoublequoteclose}\isanewline
\ \ \isacommand{assume}\isamarkupfalse%
\ {\isachardoublequoteopen}valid\ b\ w{\isachardoublequoteclose}\ \isakeyword{and}\ {\isachardoublequoteopen}w\ {\isasymnoteq}\ {\isacharbrackleft}{\kern0pt}{\isacharbrackright}{\kern0pt}{\isachardoublequoteclose}\isanewline
\ \ \isacommand{hence}\isamarkupfalse%
\ {\isachardoublequoteopen}dist\ b\ {\isacharparenleft}{\kern0pt}to{\isacharunderscore}{\kern0pt}gray\ b\ w{\isacharparenright}{\kern0pt}\ {\isacharparenleft}{\kern0pt}to{\isacharunderscore}{\kern0pt}gray\ b\ {\isacharparenleft}{\kern0pt}inc\ b\ w{\isacharparenright}{\kern0pt}{\isacharparenright}{\kern0pt}\ {\isasymnoteq}\ {\isadigit{0}}{\isachardoublequoteclose}\ \isacommand{using}\isamarkupfalse%
\ gray{\isacharunderscore}{\kern0pt}empty\ \isacommand{by}\isamarkupfalse%
\ blast\isanewline
\ \ \isacommand{with}\isamarkupfalse%
\ a\ \isacommand{have}\isamarkupfalse%
\ {\isachardoublequoteopen}dist\ b\ {\isacharparenleft}{\kern0pt}to{\isacharunderscore}{\kern0pt}gray\ b\ w{\isacharparenright}{\kern0pt}\ {\isacharparenleft}{\kern0pt}to{\isacharunderscore}{\kern0pt}gray\ b\ {\isacharparenleft}{\kern0pt}inc\ b\ w{\isacharparenright}{\kern0pt}{\isacharparenright}{\kern0pt}\ {\isachargreater}{\kern0pt}\ {\isadigit{1}}{\isachardoublequoteclose}\ \isacommand{by}\isamarkupfalse%
\ simp\isanewline
\ \ \isacommand{thus}\isamarkupfalse%
\ {\isachardoublequoteopen}False{\isachardoublequoteclose}\ \isacommand{using}\isamarkupfalse%
\ {\isacartoucheopen}valid\ b\ w{\isacartoucheclose}\ gray{\isacharunderscore}{\kern0pt}dist\ \isacommand{by}\isamarkupfalse%
\ fastforce\isanewline
\isacommand{qed}\isamarkupfalse%
%
\endisatagproof
{\isafoldproof}%
%
\isadelimproof
\isanewline
%
\endisadelimproof
\isanewline
\isacommand{lemmas}\isamarkupfalse%
\ hamming{\isacharunderscore}{\kern0pt}simps\ {\isacharequal}{\kern0pt}\ gray{\isacharunderscore}{\kern0pt}dist\ hamming{\isacharunderscore}{\kern0pt}dist\ le{\isacharunderscore}{\kern0pt}trans\ length{\isacharunderscore}{\kern0pt}gray\ length{\isacharunderscore}{\kern0pt}inc\ valid{\isacharunderscore}{\kern0pt}gray\ valid{\isacharunderscore}{\kern0pt}inc\isanewline
\isanewline
\isacommand{theorem}\isamarkupfalse%
\ gray{\isacharunderscore}{\kern0pt}hamming{\isacharcolon}{\kern0pt}\ {\isachardoublequoteopen}valid\ b\ w\ {\isasymLongrightarrow}\ hamming\ {\isacharparenleft}{\kern0pt}to{\isacharunderscore}{\kern0pt}gray\ b\ w{\isacharparenright}{\kern0pt}\ {\isacharparenleft}{\kern0pt}to{\isacharunderscore}{\kern0pt}gray\ b\ {\isacharparenleft}{\kern0pt}inc\ b\ w{\isacharparenright}{\kern0pt}{\isacharparenright}{\kern0pt}\ {\isasymle}\ {\isadigit{1}}{\isachardoublequoteclose}\isanewline
%
\isadelimproof
\ \ %
\endisadelimproof
%
\isatagproof
\isacommand{by}\isamarkupfalse%
\ {\isacharparenleft}{\kern0pt}metis\ hamming{\isacharunderscore}{\kern0pt}simps{\isacharparenright}{\kern0pt}%
\endisatagproof
{\isafoldproof}%
%
\isadelimproof
\isanewline
%
\endisadelimproof
%
\isadelimtheory
\isanewline
%
\endisadelimtheory
%
\isatagtheory
\isacommand{end}\isamarkupfalse%
%
\endisatagtheory
{\isafoldtheory}%
%
\isadelimtheory
%
\endisadelimtheory
%
\end{isabellebody}%
\endinput
%:%file=Non_Boolean_Gray.tex%:%
%:%11=6%:%
%:%27=8%:%
%:%28=8%:%
%:%29=9%:%
%:%30=10%:%
%:%39=13%:%
%:%40=14%:%
%:%41=15%:%
%:%42=16%:%
%:%43=17%:%
%:%45=20%:%
%:%46=20%:%
%:%47=21%:%
%:%48=22%:%
%:%49=23%:%
%:%50=24%:%
%:%51=24%:%
%:%52=25%:%
%:%53=26%:%
%:%60=29%:%
%:%72=32%:%
%:%73=33%:%
%:%75=36%:%
%:%76=36%:%
%:%77=37%:%
%:%80=38%:%
%:%84=38%:%
%:%85=38%:%
%:%86=39%:%
%:%87=39%:%
%:%92=39%:%
%:%95=40%:%
%:%96=41%:%
%:%97=41%:%
%:%98=42%:%
%:%101=43%:%
%:%105=43%:%
%:%106=43%:%
%:%107=44%:%
%:%108=44%:%
%:%117=47%:%
%:%119=50%:%
%:%120=50%:%
%:%121=51%:%
%:%128=52%:%
%:%129=52%:%
%:%130=53%:%
%:%131=53%:%
%:%132=53%:%
%:%133=53%:%
%:%134=54%:%
%:%135=54%:%
%:%136=55%:%
%:%137=55%:%
%:%138=56%:%
%:%139=56%:%
%:%140=56%:%
%:%141=57%:%
%:%142=57%:%
%:%143=58%:%
%:%144=58%:%
%:%145=59%:%
%:%146=59%:%
%:%147=59%:%
%:%148=60%:%
%:%149=60%:%
%:%150=60%:%
%:%151=60%:%
%:%152=60%:%
%:%153=61%:%
%:%154=61%:%
%:%155=61%:%
%:%156=61%:%
%:%157=61%:%
%:%158=62%:%
%:%159=62%:%
%:%160=62%:%
%:%161=62%:%
%:%162=62%:%
%:%163=63%:%
%:%169=63%:%
%:%172=64%:%
%:%173=65%:%
%:%174=65%:%
%:%175=66%:%
%:%178=67%:%
%:%182=67%:%
%:%183=67%:%
%:%184=68%:%
%:%185=68%:%
%:%194=71%:%
%:%195=72%:%
%:%197=75%:%
%:%198=75%:%
%:%199=76%:%
%:%206=77%:%
%:%207=77%:%
%:%208=78%:%
%:%209=78%:%
%:%210=79%:%
%:%211=79%:%
%:%212=80%:%
%:%213=80%:%
%:%214=80%:%
%:%215=81%:%
%:%216=81%:%
%:%217=81%:%
%:%218=81%:%
%:%219=82%:%
%:%225=82%:%
%:%228=83%:%
%:%229=84%:%
%:%230=84%:%
%:%233=85%:%
%:%237=85%:%
%:%238=85%:%
%:%243=85%:%
%:%246=86%:%
%:%247=87%:%
%:%248=87%:%
%:%249=88%:%
%:%256=89%:%
%:%257=89%:%
%:%258=90%:%
%:%259=90%:%
%:%260=90%:%
%:%261=90%:%
%:%262=91%:%
%:%263=91%:%
%:%264=92%:%
%:%265=92%:%
%:%266=93%:%
%:%267=93%:%
%:%268=93%:%
%:%269=93%:%
%:%270=94%:%
%:%271=94%:%
%:%272=94%:%
%:%273=94%:%
%:%274=95%:%
%:%275=95%:%
%:%276=95%:%
%:%277=96%:%
%:%278=96%:%
%:%279=97%:%
%:%280=97%:%
%:%281=97%:%
%:%282=98%:%
%:%283=98%:%
%:%284=98%:%
%:%285=98%:%
%:%286=99%:%
%:%287=99%:%
%:%288=100%:%
%:%289=100%:%
%:%290=101%:%
%:%291=101%:%
%:%292=101%:%
%:%293=102%:%
%:%294=102%:%
%:%295=102%:%
%:%296=103%:%
%:%297=103%:%
%:%298=104%:%
%:%299=104%:%
%:%300=105%:%
%:%301=105%:%
%:%302=105%:%
%:%303=106%:%
%:%304=106%:%
%:%305=107%:%
%:%306=107%:%
%:%307=108%:%
%:%308=108%:%
%:%309=109%:%
%:%310=109%:%
%:%311=110%:%
%:%312=110%:%
%:%313=111%:%
%:%314=111%:%
%:%315=112%:%
%:%316=112%:%
%:%317=112%:%
%:%318=112%:%
%:%319=113%:%
%:%320=113%:%
%:%321=114%:%
%:%322=115%:%
%:%323=115%:%
%:%324=116%:%
%:%325=116%:%
%:%326=116%:%
%:%327=117%:%
%:%328=118%:%
%:%329=118%:%
%:%330=118%:%
%:%331=119%:%
%:%332=119%:%
%:%333=119%:%
%:%334=120%:%
%:%335=121%:%
%:%336=121%:%
%:%337=121%:%
%:%338=122%:%
%:%339=122%:%
%:%340=122%:%
%:%341=123%:%
%:%342=124%:%
%:%343=124%:%
%:%344=124%:%
%:%345=125%:%
%:%346=125%:%
%:%347=125%:%
%:%348=126%:%
%:%349=127%:%
%:%350=127%:%
%:%351=127%:%
%:%352=128%:%
%:%353=128%:%
%:%354=128%:%
%:%357=131%:%
%:%358=132%:%
%:%359=132%:%
%:%360=132%:%
%:%361=133%:%
%:%362=133%:%
%:%363=133%:%
%:%366=136%:%
%:%367=137%:%
%:%368=137%:%
%:%369=137%:%
%:%370=138%:%
%:%371=138%:%
%:%372=138%:%
%:%375=141%:%
%:%376=142%:%
%:%377=142%:%
%:%378=142%:%
%:%379=143%:%
%:%380=143%:%
%:%381=143%:%
%:%384=146%:%
%:%385=147%:%
%:%386=147%:%
%:%387=147%:%
%:%388=148%:%
%:%389=148%:%
%:%390=148%:%
%:%393=151%:%
%:%394=152%:%
%:%395=152%:%
%:%396=152%:%
%:%397=153%:%
%:%398=153%:%
%:%399=153%:%
%:%401=155%:%
%:%402=156%:%
%:%403=156%:%
%:%404=156%:%
%:%405=157%:%
%:%406=157%:%
%:%407=157%:%
%:%408=157%:%
%:%409=157%:%
%:%410=158%:%
%:%411=158%:%
%:%412=158%:%
%:%413=158%:%
%:%414=159%:%
%:%415=159%:%
%:%416=160%:%
%:%422=160%:%
%:%425=161%:%
%:%426=162%:%
%:%427=162%:%
%:%428=163%:%
%:%429=164%:%
%:%430=164%:%
%:%431=165%:%
%:%434=166%:%
%:%438=166%:%
%:%439=166%:%
%:%448=169%:%
%:%449=170%:%
%:%450=171%:%
%:%451=172%:%
%:%452=173%:%
%:%453=174%:%
%:%454=175%:%
%:%456=178%:%
%:%457=178%:%
%:%458=179%:%
%:%465=180%:%
%:%466=180%:%
%:%467=181%:%
%:%468=181%:%
%:%469=182%:%
%:%470=182%:%
%:%471=183%:%
%:%472=183%:%
%:%473=183%:%
%:%474=183%:%
%:%475=184%:%
%:%476=184%:%
%:%477=184%:%
%:%478=184%:%
%:%479=185%:%
%:%480=185%:%
%:%481=185%:%
%:%482=185%:%
%:%483=186%:%
%:%489=186%:%
%:%492=187%:%
%:%493=188%:%
%:%494=188%:%
%:%495=189%:%
%:%496=190%:%
%:%497=190%:%
%:%500=191%:%
%:%504=191%:%
%:%505=191%:%
%:%510=191%:%
%:%515=192%:%
%:%520=193%:%



% optional bibliography
%\bibliographystyle{abbrv}
%\bibliography{root}

\end{document}

%%% Local Variables:
%%% mode: latex
%%% TeX-master: t
%%% End:
